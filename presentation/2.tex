\section{Többváltozós analízis}

\subsection{Alapfogalmak}

\begin{frame}
  \frametitle{Többváltozós analízis}
  \framesubtitle{Többváltozós függvények}

  \begin{block}{$\rvec f: \mathbb R^n \rightarrow \mathbb R^k$ függvény általános alakja}
    \def\arraystretch{1.2}
    \[
      \begin{bmatrix}
        x_1 \\ x_2 \\ \vdots \\ x_n
      \end{bmatrix}
      \mapsto
      \rvec f (x_1; x_2; \dots; x_n)
      :=
      \begin{bmatrix}
        f_1 (x_1; x_2; \dots; x_n) \\
        f_2 (x_1; x_2; \dots; x_n) \\
        \vdots                     \\
        f_k (x_1; x_2; \dots; x_n) \\
      \end{bmatrix}
    \]
    $\mathbb R^n$ -- értelmezési tartomány, \hfill
    $\mathbb R^k$ -- értékkészlet, \hfill
    $f_k$ -- komponens függvények.
  \end{block}

  \begin{block}{Gömbkörnyezet}
    Legyen $\rvec p \in \mathbb R^n$, ekkor a $\rvec p$ pont $\varepsilon$
    sugarú (nyílt) gömbkörnyezetén a
    \[
      B_\varepsilon(\rvec p) :=
      \Big\{\;
      \rvec x \in \mathbb R^n \;\Big|\; |\rvec x − \rvec p| < \varepsilon
      \;\Big\}\Big.
    \]
    halmazt értjük.
  \end{block}
\end{frame}

\subsection{Határérték számítás}

\begin{frame}
  \frametitle{Többváltozós analízis}
  \framesubtitle{Határérték számítás I}

  \begin{block}{Határérték}
    Legyen $\rvec f: \mathbb R^n \rightarrow \mathbb R^k$ leképezés. Azt
    mondjuk, hogy az $\rvec f$ Határértéke az $\rvec a \in \mathbb R^n$ pontban
    $\rvec A \in \mathbb R^k$, ha $\rvec A$ tetszőleges $\varepsilon$ sugarú
    gömbkörnyezetéhez létezik $\rvec a$-nak olyan $\sigma(\varepsilon)$ sugarú
    gömbkörnyezete, hogy:
    \[
      \rvec f \left(
      B_{\sigma(\varepsilon)}(\rvec a))
      \right)
      \subset
      B_\varepsilon(\rvec A)
      \text.
    \]
  \end{block}

  \begin{block}{Folytonosság}
    Azt mondjuk, hogy az $\rvec f: \mathbb R^n \rightarrow \mathbb R^k$
    leképezés folytonos az értelmezési tartományának egy belső $\rvec a \in
      \mathbb R^n$ pontjában, ha az $\rvec a$ pontbeli határérték megegyezik az
    adott pontbeli függvényértékkel, vagyis
    \[
      \lim_{\rvec x \rightarrow \rvec a} \rvec f(\rvec x) = \rvec f(\rvec a)
      \text.
    \]
  \end{block}
\end{frame}

\begin{frame}
  \frametitle{Többváltozós analízis}
  \framesubtitle{Határérték számítás II}

  \begin{block}{Átviteli elv}
    Az $\rvec f: \mathbb R^n \rightarrow \mathbb R^k$ leképezés határértéke az
    $\rvec a \in \mathbb R^n$ pontban $\rvec A \in \mathbb R^k$, akkor és csak
    akkor, ha bármeny $\rvec x_n \rightarrow \rvec a$ ($n \rightarrow \infty$)
    pontsorozat esetén $\rvec f (\rvec x_n) \rightarrow \rvec A$.
  \end{block}

  \begin{block}{Görbe menti odatartás}
    Ha az $(x_o, y_0)$ pontban keressük a határértéket, akkor felírunk egy
    olyan függvénykapcsolatot, melyre $y_0 = f(x_0)$ teljesül.
  \end{block}
\end{frame}

\begin{frame}
  \frametitle{Többváltozós analízis}
  \framesubtitle{Határérték számítás III}

  \begin{block}{Odatartás sorrendisége}
    \[
      \lim_{x \rightarrow x_0}
      \lim_{y \rightarrow y_0}
      \rvec f(x; y)
      \overset{?}{=}
      \lim_{y \rightarrow y_0}
      \lim_{x \rightarrow x_0}
      \rvec f(x; y)
    \]
  \end{block}

  \begin{block}{Origóba tartás}
    \begin{itemize}
      \item Spirál mentén:
            \[
              \left.
              \begin{array}{c}
                x = r_n \cos \varphi_n \\
                y = r_n \sin \varphi_n
              \end{array}
              \right\}
              \text{ ahol }
              r_n \rightarrow 0
            \]
      \item Polinom mentén: \[
              y = m \,x^k
            \]
    \end{itemize}
  \end{block}
\end{frame}

\begin{frame}
  \frametitle{Többváltozós analízis}
  \framesubtitle{Határértékes feladat}

  \input{exercise/limit}
\end{frame}

\subsection{Deriválás}

\begin{frame}
  \frametitle{Többváltozós analízis}
  \framesubtitle{Deriválás I}

  \begin{block}{Iránymenti derivált}
    Legyen $H \subset \mathbb R^n$, nyílt halmaz, $f: H \rightarrow \mathbb R$,
    valamint $\uvec v$ egységvektor. Ha létezik a
    \[
      \lim_{\lambda \rightarrow 0^+}
      \frac{f(\rvec a - \lambda \uvec v) - f(\rvec a)}{\lambda}
    \]
    határérték, és egy valós szám, akkor azt az $f$ függvény $\rvec a$ pontbeli,
    $\uvec v$ irányú iránymenti deriváltjának nevezzük.
    Jele: $\partial_{\uvec v} f(\rvec a)$.
  \end{block}

  \begin{block}{Parciális derivált}
    Olyan speciális iránymenti derivált, ahol az irány valamelyik bázis. Jele:
    \[
      \pdv{f(\rvec x)}{x_i}
      \text{, vagy }
      \partial_{x_i} f(\rvec x)
    \]
  \end{block}
\end{frame}

\begin{frame}
  \frametitle{Többváltozós analízis}
  \framesubtitle{Deriválás II}

  \begin{block}{Gradiens}
    Az $f: \mathbb R^n \rightarrow \mathbb R$ függvény gradiense:
    \def\arraystretch{1.4}
    \newcommand\ds{\displaystyle}
    \[
      \grad f(\rvec x) = \begin{bmatrix}
        \partial_{x_1} f(\rvec x) \\
        \partial_{x_2} f(\rvec x) \\
        \vdots                    \\
        \partial_{x_n} f(\rvec x) \\
      \end{bmatrix} = \begin{pmatrix}
        \ds\pdv{f(\rvec x)}{x_1} &
        \ds\pdv{f(\rvec x)}{x_2} &
        \hdots                   &
        \ds\pdv{f(\rvec x)}{x_n}
      \end{pmatrix}^{\mathsf T}
    \]

    A gradiens egy adott pontban megadja a függvény legnagyobb növekedés
    irányába mutató irányvektort.
  \end{block}
\end{frame}

\begin{frame}
  \frametitle{Többváltozós analízis}
  \framesubtitle{Deriválás III}

  \begin{block}{Jacobi mátrix}
    Az $\rvec f: \mathbb R^n \rightarrow \mathbb R^k$ függvény Jacobi-mátrixa:
    \def\arraystretch{1.4}
    \newcommand\ds{\displaystyle}
    \[
      \rmat J_{\rvec f(\rvec x)}
      =
      \begin{bmatrix}
        \grad^{\mathsf T} f_1(\rvec x) \\
        \grad^{\mathsf T} f_2(\rvec x) \\
        \vdots                         \\
        \grad^{\mathsf T} f_k(\rvec x) \\
      \end{bmatrix}
      =
      \begin{bmatrix}
        \partial_1 f_1 & \partial_2 f_1 & \hdots & \partial_n f_1 \\
        \partial_1 f_2 & \partial_2 f_2 & \hdots & \partial_n f_2 \\
        \vdots         & \vdots         & \ddots & \vdots         \\
        \partial_1 f_k & \partial_2 f_k & \hdots & \partial_n f_k \\
      \end{bmatrix}
    \]
  \end{block}
\end{frame}

\begin{frame}
  \frametitle{Többváltozós analízis}
  \framesubtitle{Deriválás IV}

  \begin{block}{Iránymenti derivált számítása a gradiens segítségével}
    Első derivált:
    \[
      \pdv{f}{\uvec e}
      = \uvec e \cdot \grad f
      = e_1 \pdv{f}{x_1} + e_2 \pdv{f}{x_2} + \dots + e_n \pdv{f}{x_n}
    \]
    Magasabb rendű derivált:
    \[
      \pdv[order={n+1}]{f}{\uvec e}
      = \uvec e \cdot \grad \pdv[order={n}]{f}{\uvec e}
    \]
  \end{block}
\end{frame}

\begin{frame}
  \frametitle{Többváltozós analízis}
  \framesubtitle{Deriváltós feladat}

  \input{exercise/derivative}
\end{frame}

\begin{frame}
  \frametitle{Többváltozós analízis}
  \framesubtitle{Derivált geometriai jelentése I}

  \begin{block}{Adott iránybeli érintő egyenes}
    Az $f: \mathbb R^2 \rightarrow \mathbb R$ függvény $\uvec e$ irányú
    érintőjének egyenlete az $\rvec x_0 = (x_0; y_0)$ pontban:
    \[
      \frac{x - x_0}{e_x} =
      \frac{y - y_0}{e_y} =
      \frac{z - f(\rvec x_0)}{\partial_{\uvec e} f(\rvec x_0)}
    \]
  \end{block}
\end{frame}

\begin{frame}
  \frametitle{Többváltozós analízis}
  \framesubtitle{Derivált geometriai jelentése II}

  \begin{block}{Felületi normális}
    \newcommand\ds{\displaystyle}
    Az $f: \mathbb R^2 \rightarrow \mathbb R$ függvény felületi normálisa
    adott $\rvec x_0$ pontban:
    \[
      \rvec n_\text{be} = \begin{pmatrix}
        \ds\pdv{f(\rvec x_0)}{x} & \ds\pdv{f(\rvec x_0)}{y} & -1
      \end{pmatrix}^{\mathsf T}
      \qquad
      \rvec n_\text{ki} = \begin{pmatrix}
        -\ds\pdv{f(\rvec x_0)}{x} & -\ds\pdv{f(\rvec x_0)}{y} & 1
      \end{pmatrix}^{\mathsf T}
    \]
  \end{block}

  \begin{block}{Érintősík}
    Az $f: \mathbb R^2 \rightarrow \mathbb R$ függvény érintősíkja
    adott $\rvec x_0 = (x_0; y_0)$ pontban:
    \[
      \pdv{f(\rvec x_0)}{x} (x - x_0) +
      \pdv{f(\rvec x_0)}{y} (y - y_0) =
      z - f(\rvec x_0)
    \]
  \end{block}
\end{frame}

\begin{frame}
  \frametitle{Többváltozós analízis}
  \framesubtitle{Derivált alkalmazásos feladat}

  \input{exercise/tangent}
\end{frame}

\subsection{Szélsőérték számítás}

\begin{frame}
  \frametitle{Többváltozós analízis}
  \framesubtitle{Szélsőérték számítás I}

  \begin{block}{Stacionárius pont -- szélsőérték létezésének szükséges feltétele}
    Legyen $f : H \subset \mathbb R^n \rightarrow R$, $\rvec x_0 \in \mathrm{int}\, H$,
    valamint $f$ parciális deriváltjai léteznek az $\rvec x_0$-ban, továbbá
    $\partial_i f(\rvec x_0) = 0$ bármely $i \in {1; 2; \dots; n}$. Ekkor
    $\rvec x_0$-t stacionárius pontnak hívjuk.
  \end{block}

  \begin{block}{Szélsőérték létezésének elégséges feltétele}
    Hesse-mátrix felírása:
    \def\arraystretch{1.25}
    \[
      \rmat H = \begin{bmatrix}
        \partial^2_1 f          & \partial_2 \partial_1 f & \hdots & \partial_n \partial_1 f \\
        \partial_1 \partial_2 f & \partial^2_2 f          & \hdots & \partial_n \partial_2 f \\
        \vdots                  & \vdots                  & \ddots & \vdots                  \\
        \partial_1 \partial_n f & \partial_2 \partial_n f & \hdots & \partial^2_n f          \\
      \end{bmatrix}
    \]
    $\det \rmat H > 0$ -- lok. szé.
    \hfill
    $\det \rmat H = 0$ -- ?
    \hfill
    $\det \rmat H < 0$ -- nincs szé.
  \end{block}
\end{frame}

\begin{frame}
  \frametitle{Többváltozós analízis}
  \framesubtitle{Szélsőérték számítás II}

  \begin{block}{Szélsőérték típusa a Hesse-mátrix alapján}
    Képezzük a Hesse mátrix aldeterminánsait:
    \def\arraystretch{1.25}
    \[
      H_1 = \begin{vmatrix}
        \partial^2 f
      \end{vmatrix}
      \text,
      \rmat \qquad
      H_2 = \begin{vmatrix}
        \partial^2_1 f          & \partial_2 \partial_1 f \\
        \partial_1 \partial_2 f & \partial^2_2 f
      \end{vmatrix}
      \text,
      \qquad
      H_3 = \begin{vmatrix}
        \partial^2_1 f          & \partial_2 \partial_1 f & \partial_3 \partial_1 f \\
        \partial_1 \partial_2 f & \partial^2_2 f          & \partial_3 \partial_2 f \\
        \partial_1 \partial_3 f & \partial_2 \partial_3 f & \partial^2_3 f
      \end{vmatrix}
    \]

    \begin{itemize}
      \item Lokális minimum -- ha mindegyik aldetermináns pozitív
      \item Lokális maximum -- ha váltakozó előjelűek, valamint $H_1 < 0$
    \end{itemize}
  \end{block}

  \input{exercise/stationary}
\end{frame}

\begin{frame}
  \frametitle{Többváltozós analízis}
  \framesubtitle{Szélsőérték számítás III}

  \begin{block}{Feltételes szélsőérték, Lagrange-féle multiplikátor}
    Egy $f$ függvénynek keressük a $g$ görbe mentén keressük a szélsőértékeit:
    A feladatot ekkor
    \[
      F = f + \lambda g\text{-re}
    \]
    kell megoldanunk.
  \end{block}

  \begin{exercise}{%
    Határozzuk meg az alábbi függvény abszolút szélsőértékeit az adott halmazokon!
  }
  \[
    f(x; y) = 3x + 4y
  \]
  \begin{enumerate}[a)]
    \item $H = {(x;y) : x^2 + y^2 \leq 4}$
    \item $H = {(x;y) : x^2 + y^2 \leq 4, y \geq 0}$
  \end{enumerate}

  \exsol[20.75cm]{
    \begin{enumerate}[a)]
      \item $H = {(x;y) : x^2 + y^2 \leq 4}$\\[2mm]
            %
            Gondolkodós megoldás: A gradiens kijelöli számunkra a függvény
            legnagyobb növekményének irányát:
            \[
              \grad f(x;y) = \begin{bmatrix}
                3 \\ 4
              \end{bmatrix}
              \text.
            \]
            Normáljuk a függvény gradiensét:
            \[
              \uvec v = \frac{1}{\sqrt{3^2 + 4^2}} \begin{bmatrix}
                3 \\ 4
              \end{bmatrix} = \begin{bmatrix}
                3/5 \\ 4/5
              \end{bmatrix}
              \text.
            \]
            A szélsőértékeket a 2 egység sugarú körön belül keressük. Mivel a
            függvény képe egy sík ezért a teljes értelmezési tartományán
            szélsőértékei nincsenek. A szélsőértékek ezért biztosan a tartomány
            peremén fognak elhelyezkedni. A maximum hely helyvektora:
            $2 \uvec v$, a minimum helyé pedig: $-2 \uvec v$. Az adott halmazon
            vett maximum és minimum:
            \[
              f(6/5; 8/5) = 10
              \text,\qquad
              f(-6/5; -8/5) = -10
              \text.
            \]

            Számolósabb megoldás: Lagrange-multiplikátor segítségével.
            Ekkor az alábbi függvénynek keressük a szélsőértékeit:
            \[
              F(x; y; \lambda) = 3x + 4y + \lambda (x^2 + y^2 - 4)
            \]
            Határozzuk meg a parciális deriváltakat:
            \[
              \pdv{F}{x} = 3 + 2x \lambda = 0
              \text, \qquad
              \pdv{F}{y} = 4 + 2y \lambda = 0
              \text, \qquad
              \pdv{F}{\lambda} = x^2 + y^2 - 4 = 0
              \text.
            \]
            Az első és a második egyenlet alapján:
            \[
              x = \frac{-3}{2\lambda}
              \text, \qquad
              y = \frac{-4}{2\lambda}
              \text.
            \]
            Ezek segítségével a harmadik egyenlet:
            \[
              \frac{9}{4\lambda^2} + \frac{16}{4\lambda^2} - 16 = 0
              \quad \rightarrow \quad
              \lambda_{12} = \pm \frac{5}{4}
              \text.
            \]
            A keresett pontok tehát:
            \[
              (x_1; y_1) = (3/5; 4/5)
              \text, \qquad
              (x_2; y_2) = (-3/5; -4/5)
              \text.
            \]
            A függvényértékek pedig:
            \[
              f(6/5; 8/5) = 10
              \text,\qquad
              f(-6/5; -8/5) = -10
              \text.
            \]
            Látható, hogy ezzel a módszerrel is ugyan azt a megoldást kaptuk.

      \item $H = {(x;y) : x^2 + y^2 \leq 4, y \geq 0}$\\[2mm]
            %
            Gondolkodós megoldás: Most is a gradienst fogjuk segítségül hívni.
            Mivel a függvény képe továbbra is egy sík, ezért a szintvonalai a
            gradiensre merőleges egyenesek lesznek.
            \begin{center}
              \begin{tikzpicture}[thick]
                \draw[-to, draw=red!40!gray, ultra thick]
                (-2.5, 0) -- (2.75, 0) node[below left] {$x$};
                \draw[-to, draw=red!40!gray, ultra thick]
                (0, -2.5) -- (0, 2.75) node[below left] {$y$};

                \draw[draw=cyan!40!gray, ultra thick] (0,0) circle (2);
                \fill[gray, opacity=.2] (2,0) arc [
                    start angle=0,
                    end angle=180,
                    x radius=2cm,
                    y radius=2cm,
                  ];
                \foreach \s in {-2,-1,...,0}{
                    \draw[
                      xshift=\s*1cm,
                      gray,
                    ](-3.2,2.4) -- (3.2,-2.4);
                  }
                \foreach \s in {0,1,...,3}{
                    \draw[
                      xshift=\s*1.11cm,
                      gray,
                    ](-3.2,2.4) -- (3.2,-2.4);
                  }

                \fill (0,0) circle (.1);
                \fill (-2,0) circle (.1) node[below left] {$(-2;0)$};
                \fill (1.2,1.6) circle (.1) node[above right] {$(6/5;8/5)$};;

                \draw[ultra thick, -to, yellow!40!gray] (0,0) -- (.6,.8);
              \end{tikzpicture}
            \end{center}
            Az ábra alapján a maximum és minimum:
            \[
              f(6/5; 8/5) = 10
              \text,\qquad
              f(-2; 0) = -6
              \text.
            \]

            Számolásos megoldás: Az előző részfeladatban kapott eredmények
            közül a maximum itt is helyt áll, viszont a minimumot az adott
            egyenes és a kör metszéspontjaiban kell keresnünk. Kettő
            multiplikátort kell bevezetnünk:
            \[
              F(x; y; \lambda; \mu) = 3x + 4y + \lambda (x^2 + y^2 - 4) + \mu (y)
            \]
            Határozzuk meg a parciális deriváltakat:
            \[
              \pdv{F}{x} = 3 + 2x \lambda = 0
              \text, \quad
              \pdv{F}{x} = 4 + 2y \lambda + \mu = 0
              \text, \quad
              \pdv{F}{\lambda} = x^2 + y^2 - 4 = 0
              \text, \quad
              \pdv{F}{\mu} = y = 0
              \text.
            \]
            A lehetséges szélsőérték helyek:
            \[
              f(-2;0) = -6
              \text,
              \qquad
              f(2;0) = 6
              \text.
            \]
            A tartományon a függvény maximuma és minimuma tehát:
            \[
              f(6/5; 8/5) = 10
              \text,\qquad
              f(-2; 0) = -6
              \text.
            \]
            Látható, hogy itt is mind a két módszerrel ugyan azt a megoldást
            kaptuk eredményül.
    \end{enumerate}
  }
\end{exercise}

\end{frame}

% \begin{frame}
%   \frametitle{Többváltozós analízis}
%   \framesubtitle{Szélsőértékes feladatok}
%
% \end{frame}

\subsection{Integrálszámítás}

\begin{frame}
  \frametitle{Többváltozós analízis}
  \framesubtitle{Integrálszámításos feladatok}

  \begin{exercise}{Határozzuk meg az alábbi integrál értékét!}
  \[
    \int_0^3 \int_2^y 8xy \, \mathrm dx \, \mathrm dy
  \]

  \exsol{%
    \[
      \int_0^3 \int_2^y 8xy \, \mathrm dx \, \mathrm dy =
      \int_0^3  \Big[ 4x^2y  \Big]_2^y \, \mathrm dy =
      \int_0^3  4y^3 - 16y \, \mathrm dy =
      \Big[ y^4 - 8y^2 \Big]_0^3 =
      81 - 72 =
      9
    \]
  }
\end{exercise}

  \input{exercise/integral-area}
\end{frame}
