\section{Többváltozós analízis}

\subsection{Alapfogalmak}

\begin{frame}
  \frametitle{Többváltozós analízis}
  \framesubtitle{Többváltozós függvények}

  \begin{block}{$\rvec f: \mathbb R^n \rightarrow \mathbb R^k$ függvény általános alakja}
    \def\arraystretch{1.2}
    \[
      \begin{bmatrix}
        x_1 \\ x_2 \\ \vdots \\ x_n
      \end{bmatrix}
      \mapsto
      \rvec f (x_1; x_2; \dots; x_n)
      :=
      \begin{bmatrix}
        f_1 (x_1; x_2; \dots; x_n) \\
        f_2 (x_1; x_2; \dots; x_n) \\
        \vdots                     \\
        f_k (x_1; x_2; \dots; x_n) \\
      \end{bmatrix}
    \]
    $\mathbb R^n$ -- értelmezési tartomány, \hfill
    $\mathbb R^k$ -- értékkészlet, \hfill
    $f_k$ -- komponens függvények.
  \end{block}

  \begin{block}{Gömbkörnyezet}
    Legyen $\rvec p \in \mathbb R^n$, ekkor a $\rvec p$ pont $\varepsilon$
    sugarú (nyílt) gömbkörnyezetén a
    \[
      B_\varepsilon(\rvec p) :=
      \Big\{\;
      \rvec x \in \mathbb R^n \;\Big|\; |\rvec x − \rvec p| < \varepsilon
      \;\Big\}\Big.
    \]
    halmazt értjük.
  \end{block}
\end{frame}

\subsection{Határérték számítás}

\begin{frame}
  \frametitle{Többváltozós analízis}
  \framesubtitle{Határérték számítás I}

  \begin{block}{Határérték}
    Legyen $\rvec f: \mathbb R^n \rightarrow \mathbb R^k$ leképezés. Azt
    mondjuk, hogy az $\rvec f$ Határértéke az $\rvec a \in \mathbb R^n$ pontban
    $\rvec A \in \mathbb R^k$, ha $\rvec A$ tetszőleges $\varepsilon$ sugarú
    gömbkörnyezetéhez létezik $\rvec a$-nak olyan $\sigma(\varepsilon)$ sugarú
    gömbkörnyezete, hogy:
    \[
      \rvec f \left(
      B_{\sigma(\varepsilon)}(\rvec a))
      \right)
      \subset
      B_\varepsilon(\rvec A)
      \text.
    \]
  \end{block}

  \begin{block}{Folytonosság}
    Azt mondjuk, hogy az $\rvec f: \mathbb R^n \rightarrow \mathbb R^k$
    leképezés folytonos az értelmezési tartományának egy belső $\rvec a \in
      \mathbb R^n$ pontjában, ha az $\rvec a$ pontbeli határérték megegyezik az
    adott pontbeli függvényértékkel, vagyis
    \[
      \lim_{\rvec x \rightarrow \rvec a} \rvec f(\rvec x) = \rvec f(\rvec a)
      \text.
    \]
  \end{block}
\end{frame}

\begin{frame}
  \frametitle{Többváltozós analízis}
  \framesubtitle{Határérték számítás II}

  \begin{block}{Átviteli elv}
    Az $\rvec f: \mathbb R^n \rightarrow \mathbb R^k$ leképezés határértéke az
    $\rvec a \in \mathbb R^n$ pontban $\rvec A \in \mathbb R^k$, akkor és csak
    akkor, ha bármeny $\rvec x_n \rightarrow \rvec a$ ($n \rightarrow \infty$)
    pontsorozat esetén $\rvec f (\rvec x_n) \rightarrow \rvec A$.
  \end{block}

  \begin{block}{Görbe menti odatartás}
    Ha az $(x_o, y_0)$ pontban keressük a határértéket, akkor felírunk egy
    olyan függvénykapcsolatot, melyre $y_0 = f(x_0)$ teljesül.
  \end{block}
\end{frame}

\begin{frame}
  \frametitle{Többváltozós analízis}
  \framesubtitle{Határérték számítás III}

  \begin{block}{Odatartás sorrendisége}
    \[
      \lim_{x \rightarrow x_0}
      \lim_{y \rightarrow y_0}
      \rvec f(x; y)
      \overset{?}{=}
      \lim_{y \rightarrow y_0}
      \lim_{x \rightarrow x_0}
      \rvec f(x; y)
    \]
  \end{block}

  \begin{block}{Origóba tartás}
    \begin{itemize}
      \item Spirál mentén:
            \[
              \left.
              \begin{array}{c}
                x = r_n \cos \varphi_n \\
                y = r_n \sin \varphi_n
              \end{array}
              \right\}
              \text{ ahol }
              r_n \rightarrow 0
            \]
      \item Polinom mentén: \[
              y = m \,x^k
            \]
    \end{itemize}
  \end{block}
\end{frame}

\begin{frame}
  \frametitle{Többváltozós analízis}
  \framesubtitle{Határértékes feladat}

  \begin{exercise}{Folytonosak-e az alábbi függvények az origóban?}
  % \begin{multicols}{2}
  \begin{enumerate}[a)]
    \item $f(x; y) = \begin{cases}
              \dfrac{2\,x\,y}{\sqrt{x^2 + y^2}} & \text{ ha } (x;y) \neq (0;0) \\
              0                                 & \text{ ha } (x;y) = (0;0)
            \end{cases}$

    \item $g(x) = \begin{cases}
              \dfrac{x^3 - y^3}{y - x} & \text{ ha } (x;y) \neq (0;0) \\
              0                        & \text{ ha } (x;y) = (0;0)
            \end{cases}$
  \end{enumerate}
  % \end{multicols}

  \exsol[19.25cm]{
    \begin{enumerate}[a)]
      \item $f(x; y) = \dfrac{2\,x\,y}{\sqrt{x^2 + y^2}}$\\[2mm]
            %
            A függvény folytonos, ha az adott pontban vett határértéke
            megegyezik az adott pontbeli függvényértékkel, vagyis:
            \[
              \lim_{\rvec x \rightarrow \rvec a} f(\rvec x) = f(\rvec a)
              \text.
            \]
            Mivel az origóban vagyunk kíváncsiak $f$ határérékére, azért
            élhetünk az alábbi helyettesítéssel:
            \[
              x := r \cos\varphi
              \text,\qquad
              y := r \sin\varphi
              \text.
            \]
            Végezzük el a helyettesítéseket:
            \[
              \lim_{\rvec x \rightarrow \nvec} f(\rvec x) =
              \lim_{r \rightarrow 0} \frac{
                2 r \cos(\varphi) r \sin(\varphi)
              }{
                \sqrt{(r \cos\varphi)^2 + (r \sin\varphi)^2}
              }
              \text.
            \]
            Végezzük el az egyszerűsítéseket:
            \[
              \lim_{r \rightarrow \infty} \frac{
                2 r^2 \cos \varphi \sin \varphi
              }{
                r
              } =
              \lim_{r \rightarrow 0} 2r \cos\varphi \sin\varphi
              = 0
              \text.
            \]
            Mivel a $2 \cos \varphi \sin \varphi$ szorzat korlátos, ezért az
            adott pontbeli határérték 0, amely megegyezik az adott pontbeli
            függvényértékkel, vagyis a függvény folytonos.

      \item $g(x; y) = \dfrac{x^3 - y^3}{{y - x}}$\\[2mm]
            %
            Egyszerűsítsük a törtfüggvényt:
            \[
              \frac{x^3 - y^3}{y - x} =
              \frac{(x - y)(x^2 + xy + y^2)}{-(x - y)} =
              -(x^2 + xy + y^2)
              \text.
            \]
            A határérték tehát az origóban:
            \[
              \lim_{\rvec x \rightarrow \nvec} -(x^2 + xy + y^2)
              = -(0 + 0 + 0)
              = 0
              \text.
            \]
    \end{enumerate}
  }
\end{exercise}

\end{frame}

\subsection{Deriválás}

\begin{frame}
  \frametitle{Többváltozós analízis}
  \framesubtitle{Deriválás I}

  \begin{block}{Iránymenti derivált}
    Legyen $H \subset \mathbb R^n$, nyílt halmaz, $f: H \rightarrow \mathbb R$,
    valamint $\uvec v$ egységvektor. Ha létezik a
    \[
      \lim_{\lambda \rightarrow 0^+}
      \frac{f(\rvec a - \lambda \uvec v) - f(\rvec a)}{\lambda}
    \]
    határérték, és egy valós szám, akkor azt az $f$ függvény $\rvec a$ pontbeli,
    $\uvec v$ irányú iránymenti deriváltjának nevezzük.
    Jele: $\partial_{\uvec v} f(\rvec a)$.
  \end{block}

  \begin{block}{Parciális derivált}
    Olyan speciális iránymenti derivált, ahol az irány valamelyik bázis. Jele:
    \[
      \pdv{f(\rvec x)}{x_i}
      \text{, vagy }
      \partial_{x_i} f(\rvec x)
    \]
  \end{block}
\end{frame}

\begin{frame}
  \frametitle{Többváltozós analízis}
  \framesubtitle{Deriválás II}

  \begin{block}{Gradiens}
    Az $f: \mathbb R^n \rightarrow \mathbb R$ függvény gradiense:
    \def\arraystretch{1.4}
    \newcommand\ds{\displaystyle}
    \[
      \grad f(\rvec x) = \begin{bmatrix}
        \partial_{x_1} f(\rvec x) \\
        \partial_{x_2} f(\rvec x) \\
        \vdots                    \\
        \partial_{x_n} f(\rvec x) \\
      \end{bmatrix} = \begin{pmatrix}
        \ds\pdv{f(\rvec x)}{x_1} &
        \ds\pdv{f(\rvec x)}{x_2} &
        \hdots                   &
        \ds\pdv{f(\rvec x)}{x_n}
      \end{pmatrix}^{\mathsf T}
    \]

    A gradiens egy adott pontban megadja a függvény legnagyobb növekedés
    irányába mutató irányvektort.
  \end{block}
\end{frame}

\begin{frame}
  \frametitle{Többváltozós analízis}
  \framesubtitle{Deriválás III}

  \begin{block}{Jacobi mátrix}
    Az $\rvec f: \mathbb R^n \rightarrow \mathbb R^k$ függvény Jacobi-mátrixa:
    \def\arraystretch{1.4}
    \newcommand\ds{\displaystyle}
    \[
      \rmat J_{\rvec f(\rvec x)}
      =
      \begin{bmatrix}
        \grad^{\mathsf T} f_1(\rvec x) \\
        \grad^{\mathsf T} f_2(\rvec x) \\
        \vdots                         \\
        \grad^{\mathsf T} f_k(\rvec x) \\
      \end{bmatrix}
      =
      \begin{bmatrix}
        \partial_1 f_1 & \partial_2 f_1 & \hdots & \partial_n f_1 \\
        \partial_1 f_2 & \partial_2 f_2 & \hdots & \partial_n f_2 \\
        \vdots         & \vdots         & \ddots & \vdots         \\
        \partial_1 f_k & \partial_2 f_k & \hdots & \partial_n f_k \\
      \end{bmatrix}
    \]
  \end{block}
\end{frame}

\begin{frame}
  \frametitle{Többváltozós analízis}
  \framesubtitle{Deriválás IV}

  \begin{block}{Iránymenti derivált számítása a gradiens segítségével}
    Első derivált:
    \[
      \pdv{f}{\uvec e}
      = \uvec e \cdot \grad f
      = e_1 \pdv{f}{x_1} + e_2 \pdv{f}{x_2} + \dots + e_n \pdv{f}{x_n}
    \]
    Magasabb rendű derivált:
    \[
      \pdv[order={n+1}]{f}{\uvec e}
      = \uvec e \cdot \grad \pdv[order={n}]{f}{\uvec e}
    \]
  \end{block}
\end{frame}

\begin{frame}
  \frametitle{Többváltozós analízis}
  \framesubtitle{Derivált geometriai jelentése I}

  \begin{block}{Adott iránybeli érintő egyenes}
    Az $f: \mathbb R^2 \rightarrow \mathbb R$ függvény $\uvec e$ irányú
    érintőjének egyenlete az $\rvec x_0 = (x_0; y_0)$ pontban:
    \[
      \frac{x - x_0}{e_x} =
      \frac{y - y_0}{e_y} =
      \frac{z - f(\rvec x_0)}{\partial_{\uvec e} f(\rvec x_0)}
    \]
  \end{block}
\end{frame}

\begin{frame}
  \frametitle{Többváltozós analízis}
  \framesubtitle{Derivált geometriai jelentése II}

  \begin{block}{Felületi normális}
    \newcommand\ds{\displaystyle}
    Az $f: \mathbb R^2 \rightarrow \mathbb R$ függvény felületi normálisa
    adott $\rvec x_0$ pontban:
    \[
      \rvec n_\text{be} = \begin{pmatrix}
        \ds\pdv{f(\rvec x_0)}{x} & \ds\pdv{f(\rvec x_0)}{y} & -1
      \end{pmatrix}^{\mathsf T}
      \qquad
      \rvec n_\text{ki} = \begin{pmatrix}
        -\ds\pdv{f(\rvec x_0)}{x} & -\ds\pdv{f(\rvec x_0)}{y} & 1
      \end{pmatrix}^{\mathsf T}
    \]
  \end{block}

  \begin{block}{Érintősík}
    Az $f: \mathbb R^2 \rightarrow \mathbb R$ függvény érintősíkja
    adott $\rvec x_0 = (x_0; y_0)$ pontban:
    \[
      \pdv{f(\rvec x_0)}{x} (x - x_0) +
      \pdv{f(\rvec x_0)}{y} (y - y_0) =
      z - f(\rvec x_0)
    \]
  \end{block}
\end{frame}

\begin{frame}
  \frametitle{Többváltozós analízis}
  \framesubtitle{Deriváltós feladat}

  \begin{exercise}{%
    Válaszoljuk meg az alábbi kérdéseket az alábbi függvénnyel kapcsolatban!
  }
  \[
    \rvec f(x;y;z) = \begin{bmatrix}
      x^2 + e^{2z}  \\
      \sin xy + z^6 \\
      xyz
    \end{bmatrix}
  \]
  \begin{enumerate}[a)]
    \item Adjuk meg a függvény Jacobi-mátrixát a $(2; \pi/4; 0)$ pontban!
    \item Adjuk meg az második komponens-függvény $(1; \pi; 1)$ pontbeli
          $\rvec v = (0; 3; 4)$ irányú iránymenti deriváltját!
          % \item Adjuk meg a harmadik komponens-függvény $(1;1;1)$ pontbeli
          %       olyan irányú iránymenti deriváltját, mely irány a komponens-függvény
          %       legnagyobb növekedésének irányába mutat.
  \end{enumerate}

  \exsol[18cm]{
    Határozzuk meg az egyes komponens-függvények gradienseit paraméteresen:
    \begin{align*}
      \grad f_1 & = \left[
        \begin{array}{*{3}{X{2cm}}}
          2x & 0 & 2 e^{2z}
        \end{array}
        \right]^{\mathsf T}
      \text,
      \\
      \grad f_2 & = \left[
        \begin{array}{*{3}{X{2cm}}}
          y \cos xy & x \cos xy & 6 z^5
        \end{array}
        \right]^{\mathsf T}
      \text,
      \\
      \grad f_3 & =\left[
        \begin{array}{*{3}{X{2cm}}}
          yz & xz & xy
        \end{array}
        \right]^{\mathsf T}
      \text.
    \end{align*}

    \begin{enumerate}[a)]
      \item A Jacobi mátrix paraméteresen a gradiensek alapján:
            \[
              \rmat J = \left[
                \begin{array}{*{3}{X{2cm}}}
                  2x        & 0         & 2 e^{2z} \\
                  y \cos xy & x \cos xy & 6 z^5    \\
                  yz        & xz        & xy
                \end{array}
                \right]
              \text.
            \]
            A $(2; \pi/4; 0)$ pontban kiértékelve pedig:
            \[
              \rmat J = \left[
                \begin{array}{*{3}{X{2cm}}}
                  4 & 0 & 2       \\
                  0 & 0 & 0       \\
                  0 & 0 & \pi / 2
                \end{array}
                \right]
              \text.
            \]

      \item Az gradiens adott pontbeli értéke:
            \[
              \grad f_2(1; \pi; 1) =
              \begin{bmatrix}
                -\pi \\ -1 \\ 6
              \end{bmatrix}
              \text.
            \]
            A kijelölt irányba mutató egységvektor:
            \[
              \uvec v = \frac{1}{\sqrt{0^2 + 3^2 + 4^2}}
              \begin{bmatrix}
                0 \\ 3 \\ 4
              \end{bmatrix} =
              \begin{bmatrix}
                0 \\ 3/5 \\ 4/5
              \end{bmatrix}
              \text.
            \]
            A keresett iránymenti derivált értéke az előző két vektor skaláris
            szorzatával egyenlő, vagyis:
            \[
              \left.\pdv{f_2(x;y;z)}{\rvec v}\right|_{(1; \pi; 2)}
              = \begin{bmatrix}
                -\pi \\ -1 \\ 6
              \end{bmatrix}^{\mathsf T} \begin{bmatrix}
                0 \\ 3/5 \\ 4/5
              \end{bmatrix}
              = -3/5 + 6 \cdot 4/5
              = 4,2
              \text.
            \]

            % \item A legnagyobb növekmény irányába a gradiens vektor mutat, melynek
            %       adott pontbeli értéke:
            %       \[
            %         \grad f_3(1; 1; 1) =
            %         \begin{bmatrix}
            %           1 \\ 1 \\ 1
            %         \end{bmatrix}
            %         \text.
            %       \]
            %       A keresett derivált tehát:
            %       \[
            %         \left.\pdv{f_3(x;y;z)}{\grad f_3|_{(1;1;1)}}\right|_{(1; 1; 1)}
            %         = \begin{bmatrix}
            %           1 \\ 1 \\ 1
            %         \end{bmatrix}^{\mathsf T}\frac{1}{\sqrt 3} \begin{bmatrix}
            %           1 \\ 1\\ 1
            %         \end{bmatrix}
            %         = \sqrt 3
            %         \text.
            %       \]
    \end{enumerate}
  }
\end{exercise}

\end{frame}

\subsection{Szélsőérték számítás}

\begin{frame}
  \frametitle{Többváltozós analízis}
  \framesubtitle{Szélsőérték számítás I}

  \begin{block}{Stacionárius pont -- szélsőérték létezésének szükséges feltétele}
    Legyen $f : H \subset \mathbb R^n \rightarrow R$, $\rvec x_0 \in \mathrm{int}\, H$,
    valamint $f$ parciális deriváltjai léteznek az $\rvec x_0$-ban, továbbá
    $\partial_i f(\rvec x_0) = 0$ bármely $i \in {1; 2; \dots; n}$. Ekkor
    $\rvec x_0$-t stacionárius pontnak hívjuk.
  \end{block}

  \begin{block}{Szélsőérték létezésének elégséges feltétele}
    Hesse-mátrix felírása:
    \def\arraystretch{1.25}
    \[
      \rmat H = \begin{bmatrix}
        \partial^2_1 f          & \partial_2 \partial_1 f & \hdots & \partial_n \partial_1 f \\
        \partial_1 \partial_2 f & \partial^2_2 f          & \hdots & \partial_n \partial_2 f \\
        \vdots                  & \vdots                  & \ddots & \vdots                  \\
        \partial_1 \partial_n f & \partial_2 \partial_n f & \hdots & \partial^2_n f          \\
      \end{bmatrix}
    \]
    $\det \rmat H > 0$ -- lok. szé.
    \hfill
    $\det \rmat H = 0$ -- ?
    \hfill
    $\det \rmat H < 0$ -- nincs szé.
  \end{block}
\end{frame}

\begin{frame}
  \frametitle{Többváltozós analízis}
  \framesubtitle{Szélsőérték számítás II}

  \begin{block}{Szélsőérték típusa a Hesse-mátrix alapján}
    Képezzük a Hesse mátrix aldeterminánsait:
    \def\arraystretch{1.25}
    \[
      H_1 = \begin{vmatrix}
        \partial^2 f
      \end{vmatrix}
      \text,
      \rmat \qquad
      H_2 = \begin{vmatrix}
        \partial^2_1 f          & \partial_2 \partial_1 f \\
        \partial_1 \partial_2 f & \partial^2_2 f
      \end{vmatrix}
      \text,
      \qquad
      H_3 = \begin{vmatrix}
        \partial^2_1 f          & \partial_2 \partial_1 f & \partial_3 \partial_1 f \\
        \partial_1 \partial_2 f & \partial^2_2 f          & \partial_3 \partial_2 f \\
        \partial_1 \partial_3 f & \partial_2 \partial_3 f & \partial^2_3 f
      \end{vmatrix}
    \]

    \begin{itemize}
      \item Lokális minimum -- ha mindegyik aldetermináns pozitív
      \item Lokális maximum -- ha váltakozó előjelűek, valamint $H_1 < 0$
    \end{itemize}
  \end{block}
\end{frame}

\begin{frame}
  \frametitle{Többváltozós analízis}
  \framesubtitle{Szélsőérték számítás III}

  \begin{block}{Feltételes szélsőérték, Lagrange-féle multiplikátor}
    Egy $f$ függvénynek keressük a $g$ görbe mentén keressük a szélsőértékeit:
    A feladatot ekkor
    \[
      F = f + \lambda g\text{-re}
    \]
    kell megoldanunk.
  \end{block}
\end{frame}

\begin{frame}
  \frametitle{Többváltozós analízis}
  \framesubtitle{Szélsőértékes feladatok}

  \begin{exercise}{%
    Keressük meg az alábbi függvények stacionárius pontjait! Írjuk fel az adott
    pontbeli Hesse-mátrixokat, valamint adjuk meg a pontok típusait!
  }
  \begin{enumerate}[a)]
    \item $f(x; y) = x^2 + 2xy - y^3$
    \item $g(x; y) = -x^2 + 2xy + y^3$
  \end{enumerate}

  \exsol[21.1cm]{%
    \begin{enumerate}[a)]
      \item $f(x; y) = x^2 + 2xy - y^3$\\[2mm]
            %
            Állítsuk elő a függvény elsőrendű parciális deriváltjait:
            \[
              \pdv{f}{x} = 2x + 2y
              \text,
              \qquad
              \pdv{f}{y} = 2x - 3y^2
              \text.
            \]
            Stacionárius pont ott található, ahol a függvény összes változó
            szerinti parciális deriváltjai eltűnnek, vagyis:
            \[
              2x + 2y = 0
              \text,
              \qquad
              2x - 3y^2 = 0
              \text.
            \]
            Oldjuk meg az egyenletrendszert:
            \[
              -2y - 3y^2 = 0
              \quad \rightarrow \quad
              y(-2 - 3y) = 0
              \text.
            \]
            A megoldások:
            \[
              (x_1; y_1) = (0; 0)
              \text, \qquad
              (x_2; y_2) = (2/3; -2/3)
              \text.
            \]

            Határozzuk meg a függvény másodrendű parciális deriváltjait:
            \[
              \pdv[order=2]{f}{x} = 2 \text,
              \qquad
              \pdv[order=2]{f}{y} = -6y \text,
              \qquad
              \pdv{f}{x,y}        = 2 \text.
            \]

            A Hesse-mátrix paraméteresen:
            \[
              \rmat H = \begin{bmatrix}
                2 & 2 \\ 2 & -6y
              \end{bmatrix}
              \text.
            \]

            \begin{enumerate}[1)]
              \item Vizsgáljuk meg a mátrix determinánsát $(0; 0)$ pontban:
                    \[
                      \det \rmat H(0;0) = \begin{vmatrix}
                        2 & 2 \\ 2 & 0
                      \end{vmatrix} = -4 \text.
                    \]
                    Mivel a determináns negatív, ezért a stacionárius pont
                    egy nyeregpont.

              \item Most vizsgáljuk meg a mátrix determinánsát $(2/3; -2/3)$
                    pontban:
                    \[
                      \det \rmat H(2/3;-2/3) = \begin{vmatrix}
                        2 & 2 \\ 2 & 4
                      \end{vmatrix} = 4 \text.
                    \]
                    Mivel a determináns pozitív, ezért a stacionárius pont
                    egy lehetséges szélsőérték hely. Mivel a Hesse-mátrix
                    összes aldeterminánsa pozitív, ($\det \rmat H > 0$,
                    $h_{11} > 0$,) ezért ebben a pontban a függvénynek lokális
                    minimuma van.
            \end{enumerate}

            Összegezve tehát:
            \begin{alignat*}{9}
              f & (0  ; 0  )  &  & = 0 \quad     &  & \rightarrow \quad
              \text{nyeregpont,}
              \\
              f & (2/3; -2/3) &  & = -4/27 \quad &  & \rightarrow \quad
              \text{lokális minimum.}
            \end{alignat*}

      \item $g(x; y) = -x^2 + 2xy + y^3$\\[2mm]
            %
            Állítsuk elő a függvény elsőrendű parciális deriváltjait:
            \[
              \pdv{g}{x} = -2x + 2y
              \text,
              \qquad
              \pdv{g}{y} = 2x + 3y^2
              \text.
            \]
            Stacionárius pont ott található, ahol a függvény összes változó
            szerinti parciális deriváltjai eltűnnek, vagyis:
            \[
              -2x + 2y = 0
              \text,
              \qquad
              2x + 3y^2 = 0
              \text.
            \]
            Oldjuk meg az egyenletrendszert:
            \[
              2y + 3y^2 = 0
              \quad \rightarrow \quad
              y(2 + 3y) = 0
              \text.
            \]
            A megoldások:
            \[
              (x_1; y_1) = (0; 0)
              \text, \qquad
              (x_2; y_2) = (-2/3; -2/3)
              \text.
            \]

            Határozzuk meg a függvény másodrendű parciális deriváltjait:
            \[
              \pdv[order=2]{g}{x} = -2 \text,
              \qquad
              \pdv[order=2]{g}{y} = 6y \text,
              \qquad
              \pdv{g}{x,y}        = 2 \text.
            \]

            A Hesse-mátrix paraméteresen:
            \[
              \rmat H = \begin{bmatrix}
                -2 & 2 \\ 2 & 6y
              \end{bmatrix}
              \text.
            \]

            \begin{enumerate}[1)]
              \item Vizsgáljuk meg a mátrix determinánsát $(0; 0)$ pontban:
                    \[
                      \det \rmat H(0;0) = \begin{vmatrix}
                        -2 & 2 \\ 2 & 0
                      \end{vmatrix} = -4 \text.
                    \]
                    Mivel a determináns negatív, ezért a stacionárius pont
                    egy nyeregpont.

              \item Most vizsgáljuk meg a mátrix determinánsát $(2/3; -2/3)$
                    pontban:
                    \[
                      \det \rmat H(-2/3;-2/3) = \begin{vmatrix}
                        -2 & 2 \\ 2 & -4
                      \end{vmatrix} = 4 \text.
                    \]
                    Mivel a determináns pozitív, ezért a stacionárius pont
                    egy lehetséges szélsőérték hely. Mivel a Hesse-mátrix
                    aldeterminánsainak előjele váltakozik, ($\det \rmat H > 0$,
                    $h_{11} < 0$,) ezért ebben a pontban a függvénynek
                    lokális maximuma van.
            \end{enumerate}

            Összegezve tehát:
            \begin{alignat*}{9}
              g & (0  ; 0  )   &  & = 0 \quad    &  & \rightarrow \quad
              \text{nyeregpont,}
              \\
              g & (-2/3; -2/3) &  & = 4/27 \quad &  & \rightarrow \quad
              \text{lokális maximum.}
            \end{alignat*}

    \end{enumerate}
  }
\end{exercise}

  \begin{exercise}{%
    Határozzuk meg az alábbi függvény abszolút szélsőértékeit az adott
    halmazokon!
  }
  \[
    f(x; y) = 3x + 4y
  \]
  \begin{enumerate}[a)]
    \item $H = {(x;y) : x^2 + y^2 \leq 4}$
    \item $H = {(x;y) : x^2 + y^2 \leq 4, y \geq 0}$
  \end{enumerate}

  \exsol[20.95cm]{
    \begin{enumerate}[a)]
      \item $H = {(x;y) : x^2 + y^2 \leq 4}$\\[2mm]
            %
            Gondolkodós megoldás: A gradiens kijelöli számunkra a függvény
            legnagyobb növekményének irányát:
            \[
              \grad f(x;y) = \begin{bmatrix}
                3 \\ 4
              \end{bmatrix}
              \text.
            \]
            Normáljuk a függvény gradiensét:
            \[
              \uvec v = \frac{1}{\sqrt{3^2 + 4^2}} \begin{bmatrix}
                3 \\ 4
              \end{bmatrix} = \begin{bmatrix}
                3/5 \\ 4/5
              \end{bmatrix}
              \text.
            \]
            A szélsőértékeket a 2 egység sugarú körön belül keressük. Mivel a
            függvény képe egy sík ezért a teljes értelmezési tartományán
            szélsőértékei nincsenek. A tartományon keresett szélsőértékek ezért
            biztosan annak peremén fognak elhelyezkedni. A maximum hely
            helyvektora: $(x_\text{max}; y_\text{max}) = 2 \uvec v$, a minimum
            helyé pedig: $(x_\text{min}; y_\text{min}) = -2 \uvec v$. Az adott
            halmazon vett maximum és minimum:
            \[
              f(6/5; 8/5) = 10
              \text,\qquad
              f(-6/5; -8/5) = -10
              \text.
            \]

            Számolósabb megoldás: Lagrange-multiplikátor segítségével.
            Ekkor az alábbi függvénynek keressük a szélsőértékeit:
            \[
              F(x; y; \lambda) = 3x + 4y + \lambda (x^2 + y^2 - 4)
            \]
            Határozzuk meg a parciális deriváltakat:
            \[
              \pdv{F}{x} = 3 + 2x \lambda = 0
              \text, \qquad
              \pdv{F}{y} = 4 + 2y \lambda = 0
              \text, \qquad
              \pdv{F}{\lambda} = x^2 + y^2 - 4 = 0
              \text.
            \]
            Az első és a második egyenlet alapján:
            \[
              x = \frac{-3}{2\lambda}
              \text, \qquad
              y = \frac{-4}{2\lambda}
              \text.
            \]
            Ezek segítségével a harmadik egyenlet:
            \[
              \frac{9}{4\lambda^2} + \frac{16}{4\lambda^2} - 16 = 0
              \quad \rightarrow \quad
              \lambda_{12} = \pm \frac{5}{4}
              \text.
            \]
            A keresett pontok tehát:
            \[
              (x_1; y_1) = (3/5; 4/5)
              \text, \qquad
              (x_2; y_2) = (-3/5; -4/5)
              \text.
            \]
            A függvényértékek pedig:
            \[
              f(6/5; 8/5) = 10
              \text,\qquad
              f(-6/5; -8/5) = -10
              \text.
            \]
            Látható, hogy ezzel a módszerrel is ugyan azt a megoldást kaptuk.

      \item $H = {(x;y) : x^2 + y^2 \leq 4, y \geq 0}$\\[2mm]
            %
            Gondolkodós megoldás: Most is a gradienst fogjuk segítségül hívni.
            Mivel a függvény képe továbbra is egy sík, ezért a szintvonalai a
            gradiensre merőleges egyenesek lesznek.
            \begin{center}
              \begin{tikzpicture}[thick]
                \draw[-to, draw=red!40!gray, ultra thick]
                (-2.5, 0) -- (2.75, 0) node[below left] {$x$};
                \draw[-to, draw=red!40!gray, ultra thick]
                (0, -2.5) -- (0, 2.75) node[below left] {$y$};

                \draw[draw=cyan!40!gray, ultra thick] (0,0) circle (2);
                \fill[gray, opacity=.2] (2,0) arc [
                    start angle=0,
                    end angle=180,
                    x radius=2cm,
                    y radius=2cm,
                  ];
                \foreach \s in {-2,-1,...,0}{
                    \draw[
                      xshift=\s*1cm,
                      gray,
                    ](-3.2,2.4) -- (3.2,-2.4);
                  }
                \foreach \s in {0,1,...,3}{
                    \draw[
                      xshift=\s*1.11cm,
                      gray,
                    ](-3.2,2.4) -- (3.2,-2.4);
                  }

                \fill (0,0) circle (.1);
                \fill (-2,0) circle (.1) node[below left] {$(-2;0)$};
                \fill (1.2,1.6) circle (.1) node[above right] {$(6/5;8/5)$};;

                \draw[ultra thick, -to, yellow!40!gray] (0,0) -- (.6,.8);
              \end{tikzpicture}
            \end{center}
            Az ábra alapján a maximum és minimum:
            \[
              f(6/5; 8/5) = 10
              \text,\qquad
              f(-2; 0) = -6
              \text.
            \]

            Számolásos megoldás: Az előző részfeladatban kapott eredmények
            közül a maximum itt is helyt áll, viszont a minimumot az adott
            egyenes és a kör metszéspontjaiban kell keresnünk. Kettő
            multiplikátort kell bevezetnünk:
            \[
              F(x; y; \lambda; \mu) = 3x + 4y + \lambda (x^2 + y^2 - 4) + \mu (y)
            \]
            Határozzuk meg a parciális deriváltakat:
            \[
              \pdv{F}{x} = 3 + 2x \lambda = 0
              \text, \quad
              \pdv{F}{x} = 4 + 2y \lambda + \mu = 0
              \text, \quad
              \pdv{F}{\lambda} = x^2 + y^2 - 4 = 0
              \text, \quad
              \pdv{F}{\mu} = y = 0
              \text.
            \]
            A lehetséges szélsőérték helyek:
            \[
              f(-2;0) = -6
              \text,
              \qquad
              f(2;0) = 6
              \text.
            \]
            A tartományon a függvény maximuma és minimuma tehát:
            \[
              f(6/5; 8/5) = 10
              \text,\qquad
              f(-2; 0) = -6
              \text.
            \]
            Látható, hogy itt is mind a két módszerrel ugyan azt a megoldást
            kaptuk eredményül.
    \end{enumerate}
  }
\end{exercise}

\end{frame}

\subsection{Integrálszámítás}

\begin{frame}
  \frametitle{Többváltozós analízis}
  \framesubtitle{Integrálszámításos feladatok}

  \begin{exercise}{%
    Határozzuk meg az alábbi integrál értékét a $(0;0)$, $(2,0)$, $(2;2)$
    csúcspontokkal rendelkező, háromszög alapú tartományon!
  }
  \[
    \int_A 8xy \, \mathrm dA
  \]

  \exsol[3.15cm]{%
    \begin{minipage}[t]{6.5cm}
      A tartomány szemléltetése:
      \begin{center}
        \begin{tikzpicture}[thick]
          \draw[-to, draw=red!40!gray, ultra thick]
          (-0.5, 0) -- (2.75, 0) node[above left] {$x$};
          \draw[-to, draw=red!40!gray, ultra thick]
          (0, -0.5) -- (0, 2.75) node[below left] {$y$};

          \draw[
            ultra thick,
             fill = gray!20,
          ] (0,0) node[below left] {$(0;0)$}
          -- (2,0) node[below right] {$(2;0)$} node[midway, below] {$x = 0$}
          -- (2,2) node[above right] {$(2;2)$}
          -- cycle node[midway,above,rotate=45] {$x = y$}
          ;
        \end{tikzpicture}
      \end{center}
    \end{minipage}%
    \hfill%
    \begin{minipage}[t]{7.5cm}
      A tartomány felírása:
      \vspace{12mm}
      \[
        T = \Big\{\;
        (x;y) \;\Big|\Big.\; 0 < x < y \;\land\; 0 < y < 2
        \;\Big\}
        \text.
      \]
    \end{minipage}%
    \hfill%

    \vspace{5mm}
    Végezzük el az integrálást:
    \[
      \int_0^2 \int_0^y 8xy \, \mathrm dx \, \mathrm dy =
      \int_0^2  \Big[ 4x^2y  \Big]_0^y \, \mathrm dy =
      \int_0^3  4y^3 \, \mathrm dy =
      \Big[ y^4 \Big]_0^2 =
      2^4 - 0^4 =
      16
      \text.
    \]
  }
\end{exercise}

  \begin{exercise}{%
    Határozzuk meg az alábbi integrál értékét az $(1;-1)$ középpontú,
    $2$ egység sugarú kör által meghatározott zárt tartományon!
  }
  \[
    \int_A x^2 + y^2 \, \mathrm dA
  \]

  \exsol{%
    A keresett tartomány:
    \[
      (x - 1)^2 + (y + 1)^2 \leq 2^2
      \text.
    \]

    Térjünk át polárkoordináta-rendszerre:
    \[
      x:= 1 + r \cos \varphi
      \text,
      \qquad
      y:= -1 + r \sin \varphi
      \text.
    \]

    Az integrálási határok:
    \[
      r \in [0; 2]
      \text,
      \quad
      \varphi \in [0; 2\pi]
      \text.
    \]

    A Jacobi-mátrix determinánsa:
    \[
      \begin{vmatrix}
        \cos \varphi & -r \sin \varphi \\
        \sin \varphi & r \cos \varphi
      \end{vmatrix}
      = r \cos^2 \varphi + r \sin^2 \varphi
      = r
      \text.
    \]

    Helyettesítsünk vissza:
    \[
      \int_0^2 \int_0^{2\pi} \Big(
      (1 + r \cos \varphi)^2 +
      (-1 + r \sin \varphi)^2
      \Big) \, r \, \mathrm d\varphi \, \mathrm dr
      \text.
    \]

    Bontsuk ki a zárójeleket:
    \[
      \int_0^2 \int_0^{2\pi}
      2 r + 2 r^2 \cos \varphi - 2 r^2 \sin \varphi + r^3
      \, \mathrm d\varphi \, \mathrm dr
      \text.
    \]

    Tudjuk, hogy a $[0; 2\pi]$ intervallumon vett integrálja a szinusz
    és koszinusz függvényeknek zérus, ezért az integrál az alábbi alakra
    egyszerűsödik:
    \[
      \int_0^2 \int_0^{2\pi} 2 r + r^3 \, \mathrm d\varphi \, \mathrm dr
      \text.
    \]

    Végezzük el az integrálást:
    \[
      \int_0^2 \int_0^{2\pi} 2 r + r^3 \, \mathrm d\varphi \, \mathrm dr =
      \int_0^2 4\pi r + 2 \pi r^3 \, \mathrm dr =
      \left[ 2 \pi r^2 + \frac{\pi r^4}{2} \right] =
      8 \pi + 8 \pi =
      16 \pi
      \text.
    \]
  }
\end{exercise}

\end{frame}
