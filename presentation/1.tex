\section{Függvénysorozatok, függvénysorok}

\subsection{Alapfogalmak}

\begin{frame}
  \frametitle{Függvénysorozatok, függvénysorok}
  \framesubtitle{Függvénysorozatok I}

  \begin{block}{Függvénysorozat}
    Az $f_n : I \subset \mathbb R \rightarrow \mathbb R$ sorozatot
    függvénysorozatnak nevezzük.
  \end{block}

  \begin{block}{Konvergencia}
    Ha az $x_0 \in I$ pontban az $(f_n(x_0))$ számsorozat konvergens, akkor azt
    mondjuk, hogy az $(f_n)$ függvénysorozat konvergens $x_0$-ban. A
    konvergencia\-halmaz:
    \[
      H := \Big\{\;
      x \;|\; x \in I \;\land\; (f_n) \text{ konvergens az } x \text{-ben}
      \;\Big\}
    \]
  \end{block}
\end{frame}

\begin{frame}
  \frametitle{Függvénysorozatok, függvénysorok}
  \framesubtitle{Függvénysorozatok II}

  \begin{block}{Pontonkénti konvergencia}
    Legyen $f(x) := \displaystyle\lim_{n \rightarrow \infty} f_n(x)$, $x \in H$.
    $f$-et az $(f_n)$ függvénysorozat határfüggvényének nevezzük. Azt mondjuk,
    hogy $(f_n)$ függvénysorozat pontonként konvergál az $f$ határfüggvényhez
    $H$-n, ha bármely $\varepsilon > 0$ esetén létezik $N(\varepsilon; x)$, hogy
    $|f_n(x) − f(x)| < \varepsilon$, ha $n > N(\varepsilon; x)$.
  \end{block}

  \begin{block}{Egyenletes konvergencia}
    Azt mondjuk, hogy $(f_n)$ egyenletesen konvergens az $E \subset H$ halmazon,
    ha bármely $\varepsilon > 0$ esetén létezik $N(\varepsilon): |f_n(x) − f(x)|
      < \varepsilon$, ha $n > N(\varepsilon)$ bármely $x \in E$ esetén.
  \end{block}
\end{frame}

\begin{frame}
  \frametitle{Függvénysorozatok, függvénysorok}
  \framesubtitle{Függvénysorok I}

  \begin{block}{Függvénysor}
    Legyen $f_n : I \subset \mathbb R \rightarrow \mathbb R$ függvénysorozat,
    valamint $s_n = \sum f_n$. Az így előálló függvénysorozatot az $(f_n)$
    függvénysorozatból képzett függvénysornak hívjuk és $\sum f_n$-nel jelöljük.
  \end{block}

  \begin{block}{Konvergencia}
    A $\sum f_n$ függvénysor konvergens az $x_0$ pontban, ha az $(s_n)$
    függvénysorozat konvergens az $x_0$ pontban.

    A $\sum f_n$ függvénysor konvergens a $H \subset I$ halmazon, ha az $(s_n)$
    függvénysorozat konvergens $H$-n.

    A $\sum f_n$ függvénysor egyenletesen konvergens a $E \subset H$ halmazon, ha az
    $(s_n)$ függvénysorozat egyenletesen konvergens $E$-n.
  \end{block}
\end{frame}

\begin{frame}
  \frametitle{Függvénysorozatok, függvénysorok}
  \framesubtitle{Függvénysorok II}

  \begin{block}{Konvergencia kritériumok}
    \begin{itemize}
      \item Majoráns kritérium \hfill $\sum a_n < \sum b_n$
      \item Minoráns kritérium \hfill $\sum b_n < \sum a_n$
      \item Hányadosteszt \hfill $\lim (\sfrac{a_{n+1}}{a_{n}}) < 1$
      \item Gyökteszt \hfill $\lim \sqrt[n]{a_n} < 1$
      \item Integrál-kritérium \hfill $\sum_{n_0}^\infty f(n) \; \leftrightarrow \; \int_{n_0}^\infty f(t) \,\mathrm dt$
      \item Leibnitz \hfill $\sum (-1)^n a_n \; \leftrightarrow \; (a_n)$ mon. cs. nullsorozat
      \item Abszolút konvergencia \hfill ha $\sum |a_n|$ is konvergens
      \item Feltételes konvergencia \hfill ha konvergens, de abszolút nem az
    \end{itemize}
  \end{block}
\end{frame}

\begin{frame}
  \frametitle{Függvénysorozatok, függvénysorok}
  \framesubtitle{Függvénysoros feladatok}

  \input{exercise/power-series-convergence}
\end{frame}

\subsection{Hatványsorok}

\begin{frame}
  \frametitle{Függvénysorozatok, függvénysorok}
  \framesubtitle{Hatványsorok}

  \begin{block}{Hatványsor}
    A $\sum a_n (x - x_0)$ alakú függvénysort hatványsornak nevezzük.
    A hatványsor centruma $x_0$, $a_n$ pedig az $n$-edik együttható.
  \end{block}

  \begin{block}{Konvergencia sugár}
    Egy hatványsor konvergencia sugara az alábbi képlettel számítható:
    \[
      r = \frac{1}{\limsup \sqrt[n]{|a_n|}}
      \text,
      \quad
      \text{ vagy }
      \quad
      r = \frac{1}{\limsup{|\sfrac{a_{n+1}}{a_n}|} }
      \text.
    \]
    \begin{itemize}
      \item Az $x = x_0$ pontban biztosan konvergens,
      \item Az $|x - x_0| < r$ tartományon abszolút konvergens,
      \item Az $|x - x_0| > r$ tartományon divergens
      \item Az $|x - x_0| = r$ pontokban külön meg kell vizsgálni.
    \end{itemize}
  \end{block}
\end{frame}

\begin{frame}
  \frametitle{Függvénysorozatok, függvénysorok}
  \framesubtitle{Hatványsoros feladat}

  \input{exercise/power-series-TF}
\end{frame}

\subsection{Taylor-sorok}

\begin{frame}
  \frametitle{Függvénysorozatok, függvénysorok}
  \framesubtitle{Taylor-sorok}

  \begin{block}{Taylor-polinom}
    Legyen az $f : I \subset \mathbb R \rightarrow \mathbb R$ függvény legalább
    $n$-szer differenciálható az $x_0 \in I$ pontban. Ekkor az $f$ függvény
    $x_0$ körüli $n$-edik Taylor-polinomja:
    \[
      T_n(x)
      = \sum_{k = 0}^n
      \frac{f^{(k)}(x_0)}{k!}
      (x - x_0)^k
      \text.
    \]
  \end{block}

  \begin{block}{Taylor-sor}
    Legyen az $f : I \subset \mathbb R \rightarrow \mathbb R$ függvény
    akárhányszor differenciálható az $x_0 \in I$ pontban. Ekkor az $f$ függvény
    $x_0$ körüli Taylor-sora:
    \[
      T(x)
      = \sum_{k = 0}^\infty
      \frac{f^{(k)}(x_0)}{k!}
      (x - x_0)^k
      \text.
    \]
  \end{block}
\end{frame}

\begin{frame}
  \frametitle{Függvénysorozatok, függvénysorok}
  \framesubtitle{Taylor-soros feladatok}

  \input{exercise/taylor-series-finite}
  \input{exercise/taylor-series-infinite}
\end{frame}

\subsection{Fourier-sorok}

\begin{frame}
  \frametitle{Függvénysorozatok, függvénysorok}
  \framesubtitle{Fourier-sorok I}

  \begin{block}{Fourier-sor}
    Legyen $f : \mathbb R \rightarrow \mathbb R$ egy $2p$ szerint periodikus
    függvény, amely a $[0,2p]$ intervallumon Riemann-integrálható ($f \in
      \mathcal R [0, 2p]$). Ekkor $f$ Fourier-során az alábbi trigonometrikus
    sort értjük:
    \[
      F(x)
      = a_0
      + \sum_{k = 1}^\infty a_k \cos \left( \frac{k \pi x}{p} \right)
      + \sum_{k = 1}^\infty b_k \sin \left( \frac{k \pi x}{p} \right)
    \]
    Ha a függvény $2 \pi$ szerint periodikus:
    \[
      F(x)
      = a_0
      + \sum_{n = 1}^\infty a_n \cos (nx)
      + \sum_{n = 1}^\infty b_n \sin (nx)
    \]
  \end{block}
\end{frame}

\begin{frame}
  \frametitle{Függvénysorozatok, függvénysorok}
  \framesubtitle{Fourier-sorok II}

  \begin{block}{Fourier együtthatók számítása ($2\pi$ / $2p$ periodicitás esetén)}
    \begin{alignat*}{9}
      a_0 & =
      \frac{1}{2\pi} \int_0^{2\pi} f(x) \, \mathrm d x
          &   & a_0
          &   & =
      \frac{1}{2p} \int_0^{2p} f(x) \, \mathrm d x
      \\
      a_n & =
      \frac{1}{\pi} \int_0^{2\pi} f(x) \cos(n x) \, \mathrm d x
      \hspace{2.2cm}
          &   & a_k
          &   & =
      \frac{1}{p} \int_0^{2p} f(x) \cos \left( \frac{k \pi x}{p} \right) \, \mathrm d x
      \\
      b_n & =
      \frac{1}{\pi} \int_0^{2\pi} f(x) \sin(n x) \, \mathrm d x
          &   & b_k
          &   & =
      \frac{1}{p} \int_0^{2p} f(x) \sin \left( \frac{k \pi x}{p} \right) \, \mathrm d x
    \end{alignat*}
  \end{block}
\end{frame}

\begin{frame}
  \frametitle{Függvénysorozatok, függvénysorok}
  \framesubtitle{Fourier-soros feladatok}

  \begin{exercise}{%
    Adjuk meg meg az alábbi függvények Fourier-sorában a nemzérus együtthatók
    összegét és szorzatát!
  }
  \begin{enumerate}[a)]
    \item $f(x) = 2 \cos^3 2x$
    \item $g(x) = 4 \cos^2 x \sin 2x$
  \end{enumerate}

  \exsol[4.5cm]{%
    A függvényeket trigonometrikus átalakítások segítségével hozzuk az alábbi
    alakra:
    \[
      f(x) = a_0 + \sum_{n = 1}^{\infty} \left(
      a_n \cos nx + b_n \sin nx
      \right)
      \text.
    \]

    \tcbline

    Hasznos trigonometrikus összefüggések:
    \begin{center}
      \def\arraystretch{1.1}
      \begin{tabular}[t]{| X{5cm} X{5cm} |}
        \hline
        \multicolumn{2}{| C |}{\cos^2 x + \sin^2 x = 1}                          \\[2mm]
        \sin 2x = 2 \sin x \cos x         & \cos 2x = \cos^2 x - \sin^2 x        \\[2mm]
        \sin^2 x = \dfrac{1 - \cos 2x}{2} & \cos^2 x = \dfrac{1 + \cos 2x}{2}    \\[4mm]
        \multicolumn{2}{| C |}{\sin (t \pm s) = \sin t \cos s \pm \cos t \sin s} \\[1mm]
        \multicolumn{2}{| C |}{\cos (t \pm s) = \cos t \cos s \mp \sin t \sin s} \\[1mm]
        \multicolumn{2}{| C |}{2 \sin t \sin s = \cos(t - s) - \cos(t + s)}      \\[1mm]
        \multicolumn{2}{| C |}{2 \cos t \cos s = \cos(t - s) + \cos(t + s)}      \\[1mm]
        \multicolumn{2}{| C |}{2 \sin t \cos s = \sin(t - s) + \sin(t + s)}      \\[1mm]
        %
        \hline
      \end{tabular}
    \end{center}

    \tcbline

    \begin{enumerate}[a)]
      \item $f(x) = 4 \cos^3 2x$
            \begin{align*}
              4 \cos^3 2x
               & = 4 \cos 2x \left( \frac{1 + \cos 4x}{2} \right)
              = 2 \cos 2x + 2\cos 4x \cos 2x
              \\
               & = 2 \cos 2x + \cos 2x + \cos 6x
              = 3 \cos 2x + \cos 6x
            \end{align*}

            Az nemzérus együtthatók tehát: $a_2 = 3$ és $a_6 = 1$.

            Összegük: $a_2 + a_6 = 3 + 1 = 4$.

            Szorzatuk: $a_2 \cdot a_6 = 3 \cdot 1 = 3$.

      \item $g(x) = 2 \cos^2 x \sin 2x$
            \[
              2 \cos^2 x \sin 2x
              = 2 \left( \frac{1 + \cos 2x}{2} \right) \sin 2x
              = \sin 2x + \frac{1}{2} \sin 4x
            \]

            Az nemzérus együtthatók tehát: $b_2 = 1$ és $b_4 = 1/2$.

            Összegük: $b_2 + b_4 = 1 + 1/2 = 3/2$.

            Szorzatuk: $b_2 \cdot b_4 = 1 \cdot 1/2 = 1/2$.
    \end{enumerate}
  }
\end{exercise}

  \input{exercise/fourier-square.tex}
\end{frame}
