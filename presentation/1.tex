\section{Függvénysorozatok, függvénysorok}

\subsection{Alapfogalmak}

\begin{frame}
  \frametitle{Függvénysorozatok, függvénysorok}
  \framesubtitle{Függvénysorozatok I}

  \begin{block}{Függvénysorozat}
    Az $f_n : I \subset \mathbb R \rightarrow \mathbb R$ sorozatot
    függvénysorozatnak nevezzük.
  \end{block}

  \begin{block}{Konvergencia}
    Ha az $x_0 \in I$ pontban az $(f_n(x_0))$ számsorozat konvergens, akkor azt
    mondjuk, hogy az $(f_n)$ függvénysorozat konvergens $x_0$-ban. A
    konvergencia\-halmaz:
    \[
      H := \Big\{\;
      x \;|\; x \in I \;\land\; (f_n) \text{ konvergens az } x \text{-ben}
      \;\Big\}
    \]
  \end{block}
\end{frame}

\begin{frame}
  \frametitle{Függvénysorozatok, függvénysorok}
  \framesubtitle{Függvénysorozatok II}

  \begin{block}{Pontonkénti konvergencia}
    Legyen $f(x) := \displaystyle\lim_{n \rightarrow \infty} f_n(x)$, $x \in H$.
    $f$-et az $(f_n)$ függvénysorozat határfüggvényének nevezzük. Azt mondjuk,
    hogy $(f_n)$ függvénysorozat pontonként konvergál az $f$ határfüggvényhez
    $H$-n, ha bármely $\varepsilon > 0$ esetén létezik $N(\varepsilon; x)$, hogy
    $|f_n(x) − f(x)| < \varepsilon$, ha $n > N(\varepsilon; x)$.
  \end{block}

  \begin{block}{Egyenletes konvergencia}
    Azt mondjuk, hogy $(f_n)$ egyenletesen konvergens az $E \subset H$ halmazon,
    ha bármely $\varepsilon > 0$ esetén létezik $N(\varepsilon): |f_n(x) − f(x)|
      < \varepsilon$, ha $n > N(\varepsilon)$ bármely $x \in E$ esetén.
  \end{block}
\end{frame}

\begin{frame}
  \frametitle{Függvénysorozatok, függvénysorok}
  \framesubtitle{Függvénysorok I}

  \begin{block}{Függvénysor}
    Legyen $f_n : I \subset \mathbb R \rightarrow \mathbb R$ függvénysorozat,
    valamint $s_n = \sum f_n$. Az így előálló függvénysorozatot az $(f_n)$
    függvénysorozatból képzett függvénysornak hívjuk és $\sum f_n$-nel jelöljük.
  \end{block}

  \begin{block}{Konvergencia}
    A $\sum f_n$ függvénysor konvergens az $x_0$ pontban, ha az $(s_n)$
    függvénysorozat konvergens az $x_0$ pontban.

    A $\sum f_n$ függvénysor konvergens a $H \subset I$ halmazon, ha az $(s_n)$
    függvénysorozat konvergens $H$-n.

    A $\sum f_n$ függvénysor egyenletesen konvergens a $E \subset H$ halmazon, ha az
    $(s_n)$ függvénysorozat egyenletesen konvergens $E$-n.
  \end{block}
\end{frame}

\begin{frame}
  \frametitle{Függvénysorozatok, függvénysorok}
  \framesubtitle{Függvénysorok II}

  \begin{block}{Konvergencia kritériumok}
    \begin{itemize}
      \item Majoráns kritérium \hfill $\sum a_n < \sum b_n$
      \item Minoráns kritérium \hfill $\sum b_n < \sum a_n$
      \item Hányadosteszt \hfill $\lim (\sfrac{a_{n+1}}{a_{n}}) < 1$
      \item Gyökteszt \hfill $\lim \sqrt[n]{a_n} < 1$
      \item Integrál-kritérium \hfill $\sum_{n_0}^\infty f(n) \; \leftrightarrow \; \int_{n_0}^\infty f(t) \,\mathrm dt$
      \item Leibnitz \hfill $\sum (-1)^n a_n \; \leftrightarrow \; (a_n)$ mon. cs. nullsorozat
      \item Abszolút konvergencia \hfill ha $\sum |a_n|$ is konvergens
      \item Feltételes konvergencia \hfill ha konvergens, de abszolút nem az
    \end{itemize}
  \end{block}
\end{frame}

\begin{frame}
  \frametitle{Függvénysorozatok, függvénysorok}
  \framesubtitle{Függvénysoros feladatok}

  \begin{exercise}{%
    Konvergensek-e az alábbi függvénysorok az $x_0$ pontban?
  }
  \newcommand{\tit}[2]{\begin{tabular}{*{2}{p{3.5cm}}} #1 & #2 \end{tabular}}
  \begin{enumerate}[a)]
    \item \tit{$\displaystyle\sum_{n=0}^\infty\left(\dfrac{x - 2}{x - 4}\right)^n$}{$x_0 = 5$}
    \item \tit{$\displaystyle\sum_{n=0}^\infty\left(\dfrac{x + 3}{x - 6}\right)^n$}{$x_0 = 1$}
  \end{enumerate}

  \exsol[4cm]{
    \newcommand{\tit}[2]{\begin{tabular}{*{2}{p{3.5cm}}} #1 & #2 \end{tabular}}
    \begin{enumerate}[a)]
      \item \tit{$\displaystyle\sum_{n=0}^\infty\left(\dfrac{x - 2}{x - 4}\right)^n$}{$x_0 = 5$}

            A konvergencia feltétele, hogy:
            \[
              \left| \dfrac{x - 2}{x - 4} \right| < 1
            \]
            egyenlőtlenség teljesüljön. Vizsgáljuk meg, hogy teljesül-e ez
            $x_0 = 5$ esetén:
            \[
              \left| \dfrac{x_0 - 2}{x_0 - 4} \right| =
              \left| \dfrac{5 - 2}{5 - 4} \right| =
              \left| \dfrac{3}{1} \right| =
              3
              \not <
              1
              \text.
            \]
            Látszik, hogy a feltétel nem teljesül, vagyis a függvénysorozat a
            vizsgált pontban \textbf{nem konvergens}.

      \item \tit{$\displaystyle\sum_{n=0}^\infty\left(\dfrac{x + 3}{x - 6}\right)^n$}{$x_0 = 1$}

            A konvergencia feltétele, hogy:
            \[
              \left| \dfrac{x + 3}{x - 6} \right| < 1
            \]
            egyenlőtlenség teljesüljön. Vizsgáljuk meg, hogy teljesül-e ez
            $x_0 = 1$ esetén:
            \[
              \left| \dfrac{x_0 + 3}{x_0 - 6} \right| =
              \left| \dfrac{1 + 3}{1 - 6} \right| =
              \left| \dfrac{4}{-5} \right| =
              \frac{4}{5}
              <
              1
              \text.
            \]
            A feltétel teljesül, vagyis a függvénysorozat \textbf{konvergens}
            a vizsgált pontban.
    \end{enumerate}
  }
\end{exercise}

\end{frame}

\subsection{Hatványsorok}

\begin{frame}
  \frametitle{Függvénysorozatok, függvénysorok}
  \framesubtitle{Hatványsorok}

  \begin{block}{Hatványsor}
    A $\sum a_n (x - x_0)$ alakú függvénysort hatványsornak nevezzük.
    A hatványsor centruma $x_0$, $a_n$ pedig az $n$-edik együttható.
  \end{block}

  \begin{block}{Konvergencia sugár}
    Egy hatványsor konvergencia sugara az alábbi képlettel számítható:
    \[
      r = \frac{1}{\limsup \sqrt[n]{|a_n|}}
      \text,
      \quad
      \text{ vagy }
      \quad
      r = \frac{1}{\limsup{|\sfrac{a_{n+1}}{a_n}|} }
      \text.
    \]
    \begin{itemize}
      \item Az $x = x_0$ pontban biztosan konvergens,
      \item Az $|x - x_0| < r$ tartományon abszolút konvergens,
      \item Az $|x - x_0| > r$ tartományon divergens
      \item Az $|x - x_0| = r$ pontokban külön meg kell vizsgálni.
    \end{itemize}
  \end{block}
\end{frame}

\begin{frame}
  \frametitle{Függvénysorozatok, függvénysorok}
  \framesubtitle{Hatványsoros feladat}

  \begin{exercise}{Döntsük el, hogy az alábbi állítások közül melyik igaz, és melyik hamis}
  \begin{enumerate}[a)]
    \item Ha egy hatványsor konvergens az $x = 3$ pontban, akkor a
          konvergencia-sugara minimum $r = 3$.

    \item Egy hatványsor konvergencia-sugara $r = 4$. Ha az $x = 2$ pontban
          konvergens, akkor biztosan konvergens $x = 3$-ban is.

    \item Egy $a = 3$ körüli hatványsor konvergens az $x = 6$-ban, akkor
          az $x = 0$-ban is konvergens.

    \item Egy $a = 2$ körüli hatványsor konvergens az $x = 5$-ben, akkor
          az $x = 4$-ben is konvergens.
  \end{enumerate}

  \exsol{
    \begin{enumerate}[a)]
      \item Hamis \\[1mm]
            %
            Pl.: Ha a sorfejtés centruma $a = 2$, valamint a konvergencia-sugár
            $r = 2$, akkor a sor, bár konvergens $x = 3$-ban, a
            konvergencia-sugár csak 2.

      \item Hamis \\[1mm]
            %
            Pl. Ha a sorfejtés centruma $a = -1,5$, akkor az $x = 2$-ben
            konvergens, de az $x = 3$-ban nem az.

      \item Hamis \\[1mm]
            %
            Pl. Ha a konvergencia-sugár 3, akkor fenn állhat, hogy az
            $x = 6$-ban konvergens, de az $x = 0$-ban divergens.

      \item Igaz \\[1mm]
            %
            Hiszen az $a = 2$ és $x = 5$ pontok között lévő pontok bármelyikében
            konvergens lesz a hatványsor.
    \end{enumerate}
  }
\end{exercise}

\end{frame}

\subsection{Taylor-sorok}

\begin{frame}
  \frametitle{Függvénysorozatok, függvénysorok}
  \framesubtitle{Taylor-sorok}

  \begin{block}{Taylor-polinom}
    Legyen az $f : I \subset \mathbb R \rightarrow \mathbb R$ függvény legalább
    $n$-szer differenciálható az $x_0 \in I$ pontban. Ekkor az $f$ függvény
    $x_0$ körüli $n$-edik Taylor-polinomja:
    \[
      T_n(x)
      = \sum_{k = 0}^n
      \frac{f^{(k)}(x_0)}{k!}
      (x - x_0)^k
      \text.
    \]
  \end{block}

  \begin{block}{Taylor-sor}
    Legyen az $f : I \subset \mathbb R \rightarrow \mathbb R$ függvény
    akárhányszor differenciálható az $x_0 \in I$ pontban. Ekkor az $f$ függvény
    $x_0$ körüli Taylor-sora:
    \[
      T(x)
      = \sum_{k = 0}^\infty
      \frac{f^{(k)}(x_0)}{k!}
      (x - x_0)^k
      \text.
    \]
  \end{block}
\end{frame}

\begin{frame}
  \frametitle{Függvénysorozatok, függvénysorok}
  \framesubtitle{Taylor-soros feladatok}

  \begin{exercise}{%
    Állítsuk elő az alábbi függvények $x_0$ körüli Taylor sorát!
    Hány tagból áll a Taylor-sor? Adjuk meg a nemzérus tagok együtthatóinak
    szorzatát!
  }
  \newcommand{\tit}[2]{\begin{tabular}{*{2}{p{3.5cm}}} #1 & #2 \end{tabular}}
  \begin{enumerate}[a)]
    \item \tit{$f(x) = x^3 + 2x$,}{$x_0 = -2$}

    \item \tit{$g(x) = x^2 - 6x$,}{$x_0 = 2$}

    \item \tit{$h(x) = x^4$,}{$x_0 = 0$}
  \end{enumerate}

  \exsol[11cm]{
    \newcommand{\tit}[2]{\begin{tabular}{*{2}{p{3.5cm}}} #1 & #2 \end{tabular}}
    \begin{enumerate}[a)]
      \item \tit{$f(x) = x^3 + 2x$,}{$x_0 = -2$}\\[1mm]
            %
            Állítsuk elő a függvény deriváltjait, és értékeljük ki őket az
            $x_0 = -2$ pontban.
            \begin{alignat*}{9}
               & f(x)    &  & = x^3 + 2x \hspace{2cm} &  & f(x_0)    &  & = -12 \\
               & f'(x)   &  & = 3x^2 + 2 \hspace{2cm} &  & f'(x_0)   &  & = 14  \\
               & f''(x)  &  & = 6x \hspace{2cm}       &  & f''(x_0)  &  & = -12 \\
               & f'''(x) &  & = 6 \hspace{2cm}        &  & f'''(x_0) &  & = 6
            \end{alignat*}
            Ezek alapján a Taylor-sor:
            \begin{align*}
              T(x)
              %  & = \sum_{k = 0}^n
              % \frac{f^{(k)}(x_0)}{k!}
              % (x - x_0)^k
              % \\
               & = \frac{-12}{0!}(x + 2)^0
              + \frac{14}{1!}(x + 2)^1
              + \frac{-12}{2!}(x + 2)^2
              + \frac{6}{3!}(x + 2)^3
              \\[2mm]
               & = -12
              + 14 (x + 2)
              - 6(x + 2)^2
              + 1(x + 2)^3
              \text.
            \end{align*}
            A sor tagszáma 4, az együtthatók szorzata pedig:
            \[
              (-12) \cdot 14 \cdot (-6) \cdot 1 = 1008
              \text.
            \]

      \item \tit{$g(x) = x^2 - 6x$,}{$x_0 = 2$}\\[1mm]
            %
            Állítsuk elő a függvény deriváltjait, és értékeljük ki őket az
            $x_0 = 2$ pontban.
            \begin{alignat*}{9}
               & g(x)   &  & = x^2 - 6x \hspace{2cm} &  & g(x_0)   &  & = -8 \\
               & g'(x)  &  & = 2x - 6 \hspace{2cm}   &  & g'(x_0)  &  & = -2 \\
               & g''(x) &  & = 2 \hspace{2cm}        &  & g''(x_0) &  & = 2  \\
            \end{alignat*}
            Ezek alapján a Taylor-sor:
            \begin{align*}
              T(x)
              %  & = \sum_{k = 0}^n
              % \frac{f^{(k)}(x_0)}{k!}
              % (x - x_0)^k
              % \\
               & = \frac{-8}{0!}(x - 2)^0
              + \frac{-2}{1!}(x - 2)^1
              + \frac{2}{2!}(x - 2)^2
              \\[2mm]
               & = -8
              - 2(x - 2)
              + 1(x - 2)^2
              \text.
            \end{align*}
            A sor tagszáma 3, az együtthatók szorzata pedig:
            \[
              (-8) \cdot (-2) \cdot 1 = 16
              \text.
            \]

      \item \tit{$h(x) = x^4$,}{$x_0 = 0$}\\[1mm]
            %
            A $h(x)$ függvény már önmagában egy $x_0 = 0$ körüli Taylor-sor,
            melynek tagszáma 1, az együtthatóinak szorzata pedig szintén 1.
    \end{enumerate}
  }
\end{exercise}

  \begin{exercise}{%
    Állítsuk elő az alábbi függvények $x_0$ körüli Taylor sorát!
    Adjuk meg a nulladik, első, és $n$-edik tag együtthatóját! Mekkora a
    konvergencia-sugár?
  }
  \newcommand{\tit}[2]{\begin{tabular}{*{2}{p{3.5cm}}} #1 & #2 \end{tabular}}
  \begin{enumerate}[a)]
    \item \tit{$f(x) = \dfrac{x - 3}{x - 5}$}{$x_0 = 2$}
    \item \tit{$g(x) = e^{-3x}$}{$x_0 = -1$}
  \end{enumerate}

  \exsol[20.73cm]{
    \newcommand{\tit}[2]{\begin{tabular}{*{2}{p{3.5cm}}} #1 & #2 \end{tabular}}
    \begin{enumerate}[a)]
      \item \tit{$f(x) = \dfrac{x - 3}{x - 5}$}{$x_0 = 2$}\\[2mm]
            %
            Hozzuk a függvényt egyszerűbb alakra:
            \[
              f(x)
              = \frac{x - 3}{x - 5}
              = \frac{x - 5 + 2}{x - 5}
              = 1 + \frac{2}{x - 5}
              \text.
            \]
            Állítsuk elő a függvény deriváltjait, és értékeljük ki őket az
            $x_0 = 2$ pontban.
            \begin{alignat*}{9}
               & f(x)       &  & = 1 + \frac{2}{x-5} \hspace{2cm}     &  & f(x_0)       &  & = +\frac{1}{3}                                                 \\
               & f'(x)      &  & = \frac{-2}{(x-5)^2} \hspace{2cm}    &  & f'(x_0)      &  & = -\frac{2}{9}                                                 \\
               & f''(x)     &  & = \frac{4}{(x-5)^3}                  &  & f''(x_0)     &  & = -\frac{4}{27}                                                \\
               & f'''(x)    &  & = \frac{-12}{(x-5)^4}                &  & f'''(x_0)    &  & = -\frac{4}{27}                                                \\
              % & f''''(x)   &  & = \frac{48}{(x-5)^5}                 &  & f''''(x_0)   &  & = -\frac{16}{81}                                               \\
               & f^{(n)}(x) &  & = \frac{(-1)^n 2 \, n!}{(x-5)^{n+1}} &  & f^{(n)}(x_0) &  & = \frac{(-1)^n 2 \, n!}{(-3)^{n+1}} = \frac{-2 \, n!}{3^{n+1}}
            \end{alignat*}
            Ezek alapján a Taylor-sor:
            \begin{align*}
              T(x)
               & = \frac{1}{3 \cdot 0!}
              - \frac{2}{9 \cdot 1!} (x - 2)
              - \frac{4}{27\cdot 2!} (x - 2)^2
              + \dots
              + \frac{-2 \, n!}{3^{n+1} n!} (x - 2)^n
              \\[2mm]
               & = \frac{1}{3}
              - \frac{2}{9} (x - 2)
              - \frac{2}{27} (x - 2)^2
              + \dots
              + \frac{-2}{3^{n+1}} (x - 2)^n
              \\[2mm]
               & = \frac{1}{3}
              + \sum_{n = 1}^\infty \frac{-2}{3^{n+1}} (x - 2)^n
              \text.
            \end{align*}
            A keresett tényezők:
            \[
              a_0 = \frac{1}{3}
              \text, \qquad
              a_1 = -\frac{2}{9}
              \text, \qquad
              a_n = \frac{-2}{3^{n+1}}
              \text.
            \]
            A konvergencia-sugár:
            \[
              r
              = \frac{1}{\displaystyle\limsup_{n \rightarrow \infty} \left|
                \dfrac{a_{n+1}}{a_n}
                \right|
              }
              = \frac{1}{\displaystyle\limsup_{n \rightarrow \infty} \left|
                \dfrac{-2}{3^{n+2}} \dfrac{3^{n+1}}{-2}
                \right|
              }
              = \frac{1}{\displaystyle\lim_{n \rightarrow \infty}\dfrac{1}{3}}
              = \frac{1}{1/3}
              = 3
              \text.
            \]

      \item \tit{$g(x) = e^{-3x}$}{$x_0 = 1$}\\[2mm]
            %
            Állítsuk elő a függvény deriváltjait, és értékeljük ki őket az
            $x_0 = 1$ pontban.
            \begin{alignat*}{9}
               & f(x)       &  & = e^{-3x}                 &  & f(x_0)       &  & = e^{-3}        \\
               & f'(x)      &  & = -3xe^{-3x} \hspace{2cm} &  & f'(x_0)      &  & = -3 e^{-3}     \\
               & f''(x)     &  & = 9x^2e^{-3x}             &  & f''(x_0)     &  & = 9 e^{-3}      \\
               & f^{(n)}(x) &  & = (-3x)^n e^{-3x}         &  & f^{(n)}(x_0) &  & = (-3)^n e^{-3}
            \end{alignat*}
            Ezek alapján a Taylor-sor:
            \begin{align*}
              T(x)
               & = \frac{e^{-3}}{0!}
              + \frac{-3e^{-3}}{1!}(x - 1)
              + \frac{9e^{-3}}{2!}(x - 1)^2
              + \dots
              + \frac{(-3)^n e^{-3}}{n!} (x - 1)^n
              \\
               & = \sum_{n = 0}^\infty
              \frac{(-3)^n e^{-3}}{n!}(x - 1)^n
              \text.
            \end{align*}
            A keresett tényezők:
            \[
              a_0 = e^{-3}
              \text, \qquad
              a_1 = -3 e^{-3}
              \text, \qquad
              a_n = \frac{(-3)^n e^{-3}}{n!}
              \text.
            \]
            A konvergencia-sugár:
            \[
              r
              = \frac{1}{\displaystyle\limsup_{n \rightarrow \infty} \left|
                \dfrac{a_{n+1}}{a_n}
                \right|
              }
              = \frac{1}{\displaystyle\limsup_{n \rightarrow \infty} \left|
                \dfrac{(-3)^{n+1} \vphantom{e^{-3}}}{(n+1)!}
                \dfrac{n!}{(-3)^n \vphantom{e^{-3}}}
                \right|}
              = \frac{1}{\displaystyle\lim_{n \rightarrow \infty} \left|
                \dfrac{-3}{n+1}
                \right|}
              = \frac{1}{0}
              = \infty
              \text.
            \]
    \end{enumerate}
  }
\end{exercise}

\end{frame}

\subsection{Fourier-sorok}

\begin{frame}
  \frametitle{Függvénysorozatok, függvénysorok}
  \framesubtitle{Fourier-sorok I}

  \begin{block}{Fourier-sor}
    Legyen $f : \mathbb R \rightarrow \mathbb R$ egy $2p$ szerint periodikus
    függvény, amely a $[0,2p]$ intervallumon Riemann-integrálható ($f \in
      \mathcal R [0, 2p]$). Ekkor $f$ Fourier-során az alábbi trigonometrikus
    sort értjük:
    \[
      F(x)
      = a_0
      + \sum_{k = 1}^\infty a_k \cos \left( \frac{k \pi x}{p} \right)
      + \sum_{k = 1}^\infty b_k \sin \left( \frac{k \pi x}{p} \right)
    \]
    Ha a függvény $2 \pi$ szerint periodikus:
    \[
      F(x)
      = a_0
      + \sum_{n = 1}^\infty a_n \cos (nx)
      + \sum_{n = 1}^\infty b_n \sin (nx)
    \]
  \end{block}
\end{frame}

\begin{frame}
  \frametitle{Függvénysorozatok, függvénysorok}
  \framesubtitle{Fourier-sorok II}

  \begin{block}{Fourier együtthatók számítása ($2\pi$ / $2p$ periodicitás esetén)}
    \begin{alignat*}{9}
      a_0 & =
      \frac{1}{2\pi} \int_0^{2\pi} f(x) \, \mathrm d x
          &   & a_0
          &   & =
      \frac{1}{2p} \int_0^{2p} f(x) \, \mathrm d x
      \\
      a_n & =
      \frac{1}{\pi} \int_0^{2\pi} f(x) \cos(n x) \, \mathrm d x
      \hspace{2.2cm}
          &   & a_k
          &   & =
      \frac{1}{p} \int_0^{2p} f(x) \cos \left( \frac{k \pi x}{p} \right) \, \mathrm d x
      \\
      b_n & =
      \frac{1}{\pi} \int_0^{2\pi} f(x) \sin(n x) \, \mathrm d x
          &   & b_k
          &   & =
      \frac{1}{p} \int_0^{2p} f(x) \sin \left( \frac{k \pi x}{p} \right) \, \mathrm d x
    \end{alignat*}
  \end{block}
\end{frame}

\begin{frame}
  \frametitle{Függvénysorozatok, függvénysorok}
  \framesubtitle{Fourier-soros feladatok}

  \begin{exercise}{%
    Adjuk meg meg az alábbi függvények Fourier-sorában a nemzérus együtthatók
    összegét és szorzatát!
  }
  \begin{enumerate}[a)]
    \item $f(x) = 2 \cos^3 2x$
    \item $g(x) = 4 \cos^2 x \sin 2x$
  \end{enumerate}

  \exsol[4.5cm]{%
    A függvényeket trigonometrikus átalakítások segítségével hozzuk az alábbi
    alakra:
    \[
      f(x) = a_0 + \sum_{n = 1}^{\infty} \left(
      a_n \cos nx + b_n \sin nx
      \right)
      \text.
    \]

    \tcbline

    Hasznos trigonometrikus összefüggések:
    \begin{center}
      \def\arraystretch{1.1}
      \begin{tabular}[t]{| X{5cm} X{5cm} |}
        \hline
        \multicolumn{2}{| C |}{\cos^2 x + \sin^2 x = 1}                          \\[2mm]
        \sin 2x = 2 \sin x \cos x         & \cos 2x = \cos^2 x - \sin^2 x        \\[2mm]
        \sin^2 x = \dfrac{1 - \cos 2x}{2} & \cos^2 x = \dfrac{1 + \cos 2x}{2}    \\[4mm]
        \multicolumn{2}{| C |}{\sin (t \pm s) = \sin t \cos s \pm \cos t \sin s} \\[1mm]
        \multicolumn{2}{| C |}{\cos (t \pm s) = \cos t \cos s \mp \sin t \sin s} \\[1mm]
        \multicolumn{2}{| C |}{2 \sin t \sin s = \cos(t - s) - \cos(t + s)}      \\[1mm]
        \multicolumn{2}{| C |}{2 \cos t \cos s = \cos(t - s) + \cos(t + s)}      \\[1mm]
        \multicolumn{2}{| C |}{2 \sin t \cos s = \sin(t - s) + \sin(t + s)}      \\[1mm]
        %
        \hline
      \end{tabular}
    \end{center}

    \tcbline

    \begin{enumerate}[a)]
      \item $f(x) = 4 \cos^3 2x$
            \begin{align*}
              4 \cos^3 2x
               & = 4 \cos 2x \left( \frac{1 + \cos 4x}{2} \right)
              = 2 \cos 2x + 2\cos 4x \cos 2x
              \\
               & = 2 \cos 2x + \cos 2x + \cos 6x
              = 3 \cos 2x + \cos 6x
            \end{align*}

            Az nemzérus együtthatók tehát: $a_2 = 3$ és $a_6 = 1$.

            Összegük: $a_2 + a_6 = 3 + 1 = 4$.

            Szorzatuk: $a_2 \cdot a_6 = 3 \cdot 1 = 3$.

      \item $g(x) = 2 \cos^2 x \sin 2x$
            \[
              2 \cos^2 x \sin 2x
              = 2 \left( \frac{1 + \cos 2x}{2} \right) \sin 2x
              = \sin 2x + \frac{1}{2} \sin 4x
            \]

            Az nemzérus együtthatók tehát: $b_2 = 1$ és $b_4 = 1/2$.

            Összegük: $b_2 + b_4 = 1 + 1/2 = 3/2$.

            Szorzatuk: $b_2 \cdot b_4 = 1 \cdot 1/2 = 1/2$.
    \end{enumerate}
  }
\end{exercise}

  \begin{exercise}{%
    Állítsuk elő a $[-\pi; \pi]$ intervallumon az alábbi, $2\pi$ szerint
    periodikus függvények Fourier-sorát. Adjuk meg az összegfüggvények $9 \pi$-ben
    felvett értékét, valamint az $a_0$, $a_5$ és $b_5$ együtthatókat!
  }
  \begin{enumerate}[a)]
    \item $f(x) = \sgn x$
    \item $g(x) = 1 - 2\sgn x$
    \item $h(x) = a + b \sgn x$
  \end{enumerate}

  \exsol[20.25cm]{%
    A Fourier-sor az alábbi alakot veszi fel:
    \[
      f(x) = a_0 + \sum_{n = 1}^{\infty} \left(
      a_n \cos nx + b_n \sin nx
      \right)
      \text.
    \]

    Az egyes együtthatók az alábbi képletekkel számíthatóak:
    \begin{align*}
      a_0 & =
      \frac{1}{2\pi} \int_0^{2\pi} f(x) \, \mathrm d x
      \text,
      \\
      a_n & =
      \frac{1}{\pi} \int_0^{2\pi} f(x) \cos(n x) \, \mathrm d x
      \text,
      \\
      b_n & =
      \frac{1}{\pi} \int_0^{2\pi} f(x) \sin(n x) \, \mathrm d x
      \text.
    \end{align*}

    \tcbline

    Egyszerű eset: $f(x) = \sgn x$. \\[3mm]
    %
    Ebben az esetben a függvényünk tisztán páratlan, vagyis $a_0 = a_n = 0$.
    Csak szinuszos együtthatóink vannak. Határozzuk meg ezeket:
    \begin{align*}
      b_n
       & = \frac{1}{\pi} \int_0^{2\pi} \sgn x \sin(n x) \, \mathrm d x
      \\
       & = \frac{1}{\pi} \left(
      \int_0^{\pi} \sin(n x) \, \mathrm d x -
      \int_\pi^{2\pi} \sin(n x) \, \mathrm d x
      \right)
      \\
       & = \frac{1}{\pi} \left(
      \Bigg[ \frac{-\cos nx}{n} \Bigg]_{0}^{\pi} -
      \Bigg[ \frac{-\cos nx}{n} \Bigg]_{\pi}^{2\pi}
      \right)
      \\
       & = \frac{1}{\pi} \left(
      \frac{%
        - \cos \pi n
        + \cos 0
        + \cos 2 \pi n
        - \cos \pi n
      }{n}
      \right)
      \\
       & = \frac{2}{\pi} \left(
      \frac{1 - \cos \pi n}{n}
      \right)
      = \begin{cases}
          \sfrac{4}{n \pi} \text{, ha } n \text{ páratlan,} \\
          0 \text{, ha } n \text{ páros.}
        \end{cases}
    \end{align*}

    Vagyis a páros indexű, szinuszos együtthatók zérusak, a páratlan indexűek
    értéke pedig:
    \[
      b_n = \frac{4}{n\pi}
      \text.
    \]

    A keresett együtthatók értéke tehát:
    \[
      a_0 = 0
      \text,
      \quad
      a_5 = 0
      \text,
      \quad
      b_5 = \frac{4}{5 \pi}
      \text.
    \]

    Az összegfüggvény $9\pi$-ben felvett értéke:
    \[
      F(9 \pi) = F(\pi) = \frac{f(\pi^-) + f(\pi^+)}{2} = \frac{1-1}{2} = 0
      \text.
    \]

    \tcbline

    Bonyolultabb eset: $g(x) = 1 - 2\sgn x$. \\[3mm]
    %
    Ebben az esetben a függvény nem páratlan, viszont azzá tehető. Ha az $y$
    tengely mentén negatív irányban eltoljuk egy egységnyit. Keressük most ennek
    a $g^*(x) = -2 \sgn x$ függvénynek a Fourier együtthatóit. Mivel ez a
    függvény már páratlan, ezért csak szinuszos együtthatói lesznek:
    \begin{align*}
      b_n
       & = \frac{1}{\pi} \int_0^{2\pi} -2\sgn x \sin(n x) \, \mathrm d x
      \\
       & = \frac{-2}{\pi} \int_0^{2\pi} \sgn x \sin(n x) \, \mathrm d x
      \\
       & = \frac{-2}{\pi} \left(
      \int_0^{\pi} \sin(n x) \, \mathrm d x -
      \int_\pi^{2\pi} \sin(n x) \, \mathrm d x
      \right)
      \\
       & = \frac{-2}{\pi} \left(
      \Bigg[ \frac{-\cos nx}{n} \Bigg]_{0}^{\pi} -
      \Bigg[ \frac{-\cos nx}{n} \Bigg]_{\pi}^{2\pi}
      \right)
      \\
       & = \frac{-2}{\pi} \left(
      \frac{%
        - \cos \pi n
        + \cos 0
        + \cos 2 \pi n
        - \cos \pi n
      }{n}
      \right)
      \\
       & = \frac{-4}{\pi} \left(
      \frac{1 - \cos \pi n}{n}
      \right)
      = \begin{cases}
          \sfrac{-8}{n \pi} \text{, ha } n \text{ páratlan,} \\
          0 \text{, ha } n \text{ páros.}
        \end{cases}
    \end{align*}

    Vagyis a páros indexű, szinuszos együtthatók zérusak, a páratlan indexűek
    értéke pedig:
    \[
      b_n = \frac{-8}{n\pi}
      \text.
    \]

    Az eredeti függvény együtthatói egyetlen kivétellel megegyeznek $g^*(x)$
    függvény együtthatóival, hiszen az eltolás miatt a $g(x)$ függvénynek
    lesz egy $a_0 = 1$-es együtthatója is. A keresett együtthatók tehát:
    \[
      a_0 = 1
      \text,
      \quad
      a_5 = 0
      \text,
      \quad
      b_5 = \frac{-8}{5 \pi}
      \text.
    \]

    Az összegfüggvény $9\pi$-ben felvett értéke:
    \[
      G(9 \pi) = G(\pi) = \frac{g(\pi^-) + g(\pi^+)}{2} = \frac{-1+3}{2} = 1
      \text.
    \]

    \tcbline

    Általános eset: $h(x) = a + b \sgn x$. \\[3mm]
    %
    Ebben az esetben sem páros a függvényünk, viszont azzá tehető, hogyha
    az $y$ tengely mentén $a$ egységgel lefele toljuk el. $h^*(x) = b \sgn x$
    függvény már páratlan, tehát ennek is csak szinuszos együtthatói lesznek:
    \begin{align*}
      b_n
       & = \frac{1}{\pi} \int_0^{2\pi} b \sgn x \sin(n x) \, \mathrm d x
      \\
       & = \frac{b}{\pi} \int_0^{2\pi} \sgn x \sin(n x) \, \mathrm d x
      \\
       & = \frac{b}{\pi} \left(
      \int_0^{\pi} \sin(n x) \, \mathrm d x -
      \int_\pi^{2\pi} \sin(n x) \, \mathrm d x
      \right)
      \\
       & = \frac{b}{\pi} \left(
      \Bigg[ \frac{-\cos n x}{n} \Bigg]_{0}^{\pi} -
      \Bigg[ \frac{-\cos n x}{n} \Bigg]_{\pi}^{2\pi}
      \right)
      \\
       & = \frac{b}{\pi} \left(
      \frac{%
        - \cos \pi n
        + \cos 0
        + \cos 2 \pi n
        - \cos \pi n
      }{n}
      \right)
      \\
       & = \frac{2b}{\pi} \left(
      \frac{1 - \cos \pi n}{n}
      \right)
      = \begin{cases}
          \sfrac{4b}{n \pi} \text{, ha } n \text{ páratlan,} \\
          0 \text{, ha } n \text{ páros.}
        \end{cases}
    \end{align*}

    Vagyis a páros indexű, szinuszos együtthatók zérusak, a páratlan indexűek
    értéke pedig:
    \[
      b_n = \frac{4b}{n\pi}
      \text.
    \]

    Az eredeti függvény együtthatói egyetlen kivétellel megegyeznek $h^*(x)$
    függvény együtthatóival, hiszen az eltolás miatt a $h(x)$ függvénynek
    lesz egy $a_0 = a$-s együtthatója is. A keresett együtthatók tehát:
    \[
      a_0 = a
      \text,
      \quad
      a_5 = 0
      \text,
      \quad
      b_5 = \frac{4b}{5 \pi}
      \text.
    \]

    Az összegfüggvény $9\pi$-ben felvett értéke:
    \[
      H(9 \pi) = H(\pi) = \frac{h(\pi^-) + h(\pi^+)}{2} = \frac{(a+b)+(a-b)}{2} = a
      \text.
    \]
  }
\end{exercise}

\end{frame}
