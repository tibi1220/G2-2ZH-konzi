\begin{exercise}{%
    Adjuk meg meg az alábbi függvények Fourier-sorában a nemzérus együtthatók
    összegét és szorzatát!
  }
  \begin{enumerate}[a)]
    \item $f(x) = 2 \cos^3 2x$
    \item $g(x) = 4 \cos^2 x \sin 2x$
  \end{enumerate}

  \exsol[5.1cm]{%
    % A függvényeket trigonometrikus átalakítások segítségével hozzuk az alábbi
    % alakra:
    % \[
    %   f(x) = a_0 + \sum_{n = 1}^{\infty} \left(
    %   a_n \cos nx + b_n \sin nx
    %   \right)
    %   \text.
    % \]
    %
    % \tcbline
    %
    \begin{enumerate}[a)]
      \item $f(x) = 4 \cos^3 2x$
            \begin{align*}
              4 \cos^3 2x
               & = 4 \cos 2x \left( \frac{1 + \cos 4x}{2} \right)
              = 2 \cos 2x + 2\cos 4x \cos 2x
              \\
               & = 2 \cos 2x + \cos 2x + \cos 6x
              = 3 \cos 2x + \cos 6x
            \end{align*}

            Az nemzérus együtthatók tehát: $a_2 = 3$ és $a_6 = 1$.

            Összegük: $a_2 + a_6 = 3 + 1 = 4$.

            Szorzatuk: $a_2 \cdot a_6 = 3 \cdot 1 = 3$.

      \item $g(x) = 2 \cos^2 x \sin 2x$
            \[
              2 \cos^2 x \sin 2x
              = 2 \left( \frac{1 + \cos 2x}{2} \right) \sin 2x
              = \sin 2x + \frac{1}{2} \sin 4x
            \]

            Az nemzérus együtthatók tehát: $b_2 = 1$ és $b_4 = 1/2$.

            Összegük: $b_2 + b_4 = 1 + 1/2 = 3/2$.

            Szorzatuk: $b_2 \cdot b_4 = 1 \cdot 1/2 = 1/2$.
    \end{enumerate}
  }
\end{exercise}
