\begin{exercise}{%
    Határozzuk meg az alábbi integrál értékét az $(1;-1)$ középpontú,
    $2$ egység sugarú kör által meghatározott zárt tartományon!
  }
  \[
    \int_A x^2 + y^2 \, \mathrm dA
  \]

  \exsol{%
    A keresett tartomány:
    \[
      (x - 1)^2 + (y + 1)^2 \leq 2^2
      \text.
    \]

    Térjünk át polárkoordináta-rendszerre:
    \[
      x:= 1 + r \cos \varphi
      \text,
      \qquad
      y:= -1 + r \sin \varphi
      \text.
    \]

    Az integrálási határok:
    \[
      r \in [0; 2]
      \text,
      \quad
      \varphi \in [0; 2\pi]
      \text.
    \]

    A Jacobi-mátrix determinánsa:
    \[
      \begin{vmatrix}
        \cos \varphi & -r \sin \varphi \\
        \sin \varphi & r \cos \varphi
      \end{vmatrix}
      = r \cos^2 \varphi + r \sin^2 \varphi
      = r
      \text.
    \]

    Helyettesítsünk vissza:
    \[
      \int_0^2 \int_0^{2\pi} \Big(
      (1 + r \cos \varphi)^2 +
      (-1 + r \sin \varphi)^2
      \Big) \, r \, \mathrm d\varphi \, \mathrm dr
      \text.
    \]

    Bontsuk ki a zárójeleket:
    \[
      \int_0^2 \int_0^{2\pi}
      2 r + 2 r^2 \cos \varphi - 2 r^2 \sin \varphi + r^3
      \, \mathrm d\varphi \, \mathrm dr
      \text.
    \]

    Tudjuk, hogy a $[0; 2\pi]$ intervallumon vett integrálja a szinusz és a
    koszinusz függvényeknek is zérus, ezért az integrál az alábbi alakra
    egyszerűsödik:
    \[
      \int_0^2 \int_0^{2\pi} 2 r + r^3 \, \mathrm d\varphi \, \mathrm dr
      \text.
    \]

    Végezzük el az integrálást:
    \[
      \int_0^2 \int_0^{2\pi} 2 r + r^3 \, \mathrm d\varphi \, \mathrm dr =
      \int_0^2 4\pi r + 2 \pi r^3 \, \mathrm dr =
      \left[ 2 \pi r^2 + \frac{\pi r^4}{2} \right]_0^2 =
      8 \pi + 8 \pi =
      16 \pi
      \text.
    \]
  }
\end{exercise}
