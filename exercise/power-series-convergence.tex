\begin{exercise}{%
    Konvergensek-e az alábbi függvénysorok az $x_0$ pontban?
  }
  \newcommand{\tit}[2]{\begin{tabular}{*{2}{p{3.5cm}}} #1 & #2 \end{tabular}}
  \begin{enumerate}[a)]
    \item \tit{$\displaystyle\sum_{n=0}^\infty\left(\dfrac{x - 2}{x - 4}\right)^n$}{$x_0 = 5$}
    \item \tit{$\displaystyle\sum_{n=0}^\infty\left(\dfrac{x + 3}{x - 6}\right)^n$}{$x_0 = 1$}
  \end{enumerate}

  \exsol[7cm]{
    \newcommand{\tit}[2]{\begin{tabular}{*{2}{p{3.5cm}}} #1 & #2 \end{tabular}}
    \begin{enumerate}[a)]
      \item \tit{$\displaystyle\sum_{n=0}^\infty\left(\dfrac{x - 2}{x - 4}\right)^n$}{$x_0 = 5$}

            A konvergencia feltétele, hogy:
            \[
              \left| \dfrac{x - 2}{x - 4} \right| < 1
            \]
            egyenlőtlenség teljesüljön. Vizsgáljuk meg, hogy teljesül-e ez
            $x_0 = 5$ esetén:
            \[
              \left| \dfrac{x_0 - 2}{x_0 - 4} \right| =
              \left| \dfrac{5 - 2}{5 - 4} \right| =
              \left| \dfrac{3}{1} \right| =
              3
              \not\le
              1
              \text.
            \]
            Látszik, hogy a feltétel nem teljesül, vagyis a függvénysorozat a
            vizsgált pontban nem konvergens.

      \item \tit{$\displaystyle\sum_{n=0}^\infty\left(\dfrac{x + 3}{x - 6}\right)^n$}{$x_0 = 1$}

            A konvergencia feltétele, hogy:
            \[
              \left| \dfrac{x + 3}{x - 6} \right| < 1
            \]
            egyenlőtlenség teljesüljön. Vizsgáljuk meg, hogy teljesül-e ez
            $x_0 = 1$ esetén:
            \[
              \left| \dfrac{x_0 + 3}{x_0 - 6} \right| =
              \left| \dfrac{1 + 3}{1 - 6} \right| =
              \left| \dfrac{4}{-5} \right| =
              \frac{4}{5}
              \le
              1
              \text.
            \]
            A feltétel teljesül, vagyis a függvénysorozat konvergens a vizsgált
            pontban.
    \end{enumerate}
  }
\end{exercise}
